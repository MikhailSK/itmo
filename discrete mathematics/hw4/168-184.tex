\input{preamble.sty}

\begin{document}

\setcounter{section}{167}

\section{}
Используя теорему о рекурсии, докажите, что язык программ, которые останавливаются на пустом вводе, является неразрешимым. Является ли этот язык перечислимым?

\begin{minted}{python}
p(_):
    if r(p):
        while True:
            pass
    else:
        return
\end{minted}

Да, перечислим:
\begin{minted}[escapeinside=||,mathescape=true]{python}
for t in |$\N$|:
    for p in |$\overline{1, t}$|:
        if p(tl=t).halts():
            print(p)
\end{minted}

\section{}
Используя теорему о рекурсии, докажите, что язык программ, которые не останавливаются на пустом вводе, является неразрешимым. Является ли этот язык перечислимым?

\begin{minted}{python}
p(_):
    if r(p):
        return
    else:
        while True:
            pass
\end{minted}

Язык не перечислим, т.к. если есть если \(L \in RE\) и \(L \in coRE\), то \(L \in R\)

\section{}
Используя теорему о рекурсии, докажите, что язык программ, которые допускают бесконечное число слов, является неразрешимым.

\begin{minted}{python}
p(_):
    if r(p):
        return True
    else:
        return False
\end{minted}

\section{}
Используя теорему о рекурсии, докажите, что язык программ, которые допускают свой собственный исходный код, является неразрешимым.


\begin{minted}{python}
p(x):
    if x != p:
        return False
    if r(p):
        return False
    else:
        return True
\end{minted}

\section{}
Докажите, что существуют две различные программы $p$ и $q$, такие что программа $p$ печатает текст программы $q$, а программа $q$ печатает текст программы $p$.

\section{}
Докажите, что существует бесконечная последовательность различных программ $p_i$, такая что $p_1$ печатает пустую строку, а $p_i$ печатает текст программы $p_{i-1}$.

То же самое, но каждая строка не содержит в себе определение себя.

\section{}
Докажите, что существует бесконечная последовательность различных программ $p_i$, такая что $p_i$ печатает текст программы $p_{i+1}$.

То же самое, но в \(i\)-той строке мы определяем \(i + 1\)-ую.

\section{}
Докажите, что для любого конечного $n$ существует последовательность программ $p_1, p_2, \ldots, p_n$, что $p_i$ печатает текст $p_{i+1}$, а $p_n$ печатает текст $p_1$.

\begin{minted}{python}
quine(i):
    return f'''print(s1 = {s1}
    s2 = {s2}
    ...
    sn = {sn}
    print(s{i + 1 % n}))'''
s1 = quine(1)
s2 = quine(2)
...
sn = quine(n)
print(s1)
\end{minted}

\texttt{quine} --- макрос, а не часть исходного кода.

% Аналогично 172, но с \(n\) строк.

\section{}
Докажите, что язык программ, для которых не существует более короткой программы, которая на любом входе ведёт себя так же, является неразрешимым.

\section{}
Докажите, что язык программ, для которых не существует программы такой же длины, которая на любом входе ведёт себя так же, является либо конечным, либо неразрешимым.

\section{}
Busy Beaver. Функция $BB(n)$ возвращает длину максимальной строки, которую программа длины $n$ может вывести на пустом входе и завершиться. Докажите, что $BB$ является невычислимой.

\section{}
Докажите, что для любой всюду определенной вычислимой функции $f$ найдется значение $n$, для которого $BB(n) > f(n)$.

Соуфивается

\section{}
Докажите, что для любой всюду определенной вычислимой функции $f$ найдется бесконечно много значений $n$, для которых $BB(n) > f(n)$.

\section{}
Колмогоровская сложность. $K(s)$ это длина минимальной программы, которая на пустом входе выводит строку $s$ и завершается. Докажите, что $K$ является невычислимой.

\section{}
Пусть для любой строки $s$ выполнено $K(s) \ge f(s)$, где $f$ — всюду определенная вычислимая функция. Докажите, что найдется константа $C$, такая что $f(s) \le C$ для любой $s$.

\end{document}
