\documentclass[12pt, a4paper]{article}

%<*preamble>
% Math symbols
\usepackage{amsmath, amsthm, amsfonts, amssymb}
\usepackage{accents}
\usepackage{esvect}
\usepackage{mathrsfs}
\usepackage{mathtools}
\mathtoolsset{showonlyrefs}
\usepackage{cmll}
\usepackage{stmaryrd}
\usepackage{physics}
\usepackage[normalem]{ulem}
\usepackage{ebproof}
\usepackage{extarrows}

% Page layout
\usepackage{geometry, a4wide, parskip, fancyhdr}

% Font, encoding, russian support
\usepackage[russian]{babel}
\usepackage[sb]{libertine}
\usepackage{xltxtra}

% Listings
\usepackage{listings}
\lstset{basicstyle=\ttfamily,breaklines=true}
\setmonofont[Scale=MatchLowercase]{JetBrains Mono}

% Miscellaneous
\usepackage{array}
\usepackage{booktabs}\renewcommand{\arraystretch}{1.2}
\usepackage{calc}
\usepackage{caption}
\usepackage{subcaption}
\captionsetup{justification=centering,margin=2cm}
\usepackage{catchfilebetweentags}
\usepackage{enumitem}
\usepackage{etoolbox}
\usepackage{float}
\usepackage{lastpage}
\usepackage{minted}
\usepackage{svg}
\usepackage{wrapfig}
\usepackage{xcolor}
\usepackage[makeroom]{cancel}

\newcolumntype{L}{>{$}l<{$}}
    \newcolumntype{C}{>{$}c<{$}}
\newcolumntype{R}{>{$}r<{$}}

% Footnotes
\usepackage[hang]{footmisc}
\setlength{\footnotemargin}{2mm}
\makeatletter
\def\blfootnote{\gdef\@thefnmark{}\@footnotetext}
\makeatother

% References
\usepackage{hyperref}
\hypersetup{
    colorlinks,
    linkcolor={blue!80!black},
    citecolor={blue!80!black},
    urlcolor={blue!80!black},
}

% tikz
\usepackage{tikz}
\usepackage{tikz-cd}
\usetikzlibrary{arrows.meta}
\usetikzlibrary{decorations.pathmorphing}
\usetikzlibrary{calc}
\usetikzlibrary{patterns}
\usepackage{pgfplots}
\pgfplotsset{width=10cm,compat=1.9}
\newcommand\irregularcircle[2]{% radius, irregularity
    \pgfextra {\pgfmathsetmacro\len{(#1)+rand*(#2)}}
    +(0:\len pt)
    \foreach \a in {10,20,...,350}{
            \pgfextra {\pgfmathsetmacro\len{(#1)+rand*(#2)}}
            -- +(\a:\len pt)
        } -- cycle
}

\providetoggle{useproofs}
\settoggle{useproofs}{false}

\pagestyle{fancy}
\lhead{Лабораторная работа №2}
\lfoot{Михайлов Максим}
\rfoot{M3337}
\cfoot{}
\rhead{стр. \thepage\ из \pageref*{LastPage}}

\newcommand{\R}{\mathbb{R}}
\newcommand{\Q}{\mathbb{Q}}
\newcommand{\Z}{\mathbb{Z}}
\newcommand{\B}{\mathbb{B}}
\newcommand{\N}{\mathbb{N}}
\renewcommand{\Re}{\mathfrak{R}}
\renewcommand{\Im}{\mathfrak{I}}

\newcommand{\const}{\text{const}}
\newcommand{\cond}{\text{cond}}

\newcommand{\teormin}{\textcolor{red}{!}\ }

\DeclareMathOperator*{\xor}{\oplus}
\DeclareMathOperator*{\equ}{\sim}
\DeclareMathOperator{\sign}{\text{sign}}
\DeclareMathOperator{\Sym}{\text{Sym}}
\DeclareMathOperator{\Asym}{\text{Asym}}

\DeclarePairedDelimiter{\ceil}{\lceil}{\rceil}

% godel
\newbox\gnBoxA
\newdimen\gnCornerHgt
\setbox\gnBoxA=\hbox{$\ulcorner$}
\global\gnCornerHgt=\ht\gnBoxA
\newdimen\gnArgHgt
\def\godel #1{%
    \setbox\gnBoxA=\hbox{$#1$}%
    \gnArgHgt=\ht\gnBoxA%
    \ifnum     \gnArgHgt<\gnCornerHgt \gnArgHgt=0pt%
    \else \advance \gnArgHgt by -\gnCornerHgt%
    \fi \raise\gnArgHgt\hbox{$\ulcorner$} \box\gnBoxA %
    \raise\gnArgHgt\hbox{$\urcorner$}}

% \theoremstyle{plain}

\theoremstyle{definition}
\newtheorem{theorem}{Теорема}
\newtheorem*{definition}{Определение}
\newtheorem{axiom}{Аксиома}
\newtheorem*{axiom*}{Аксиома}
\newtheorem{lemma}{Лемма}
\newenvironment{solution}[1][Решение.]{\begin{proof}[#1]}{\end{proof}}

\theoremstyle{remark}
\newtheorem*{remark}{Примечание}
\newtheorem*{exercise}{Упражнение}
\newtheorem{corollary}{Следствие}[theorem]
\newtheorem*{statement}{Утверждение}
\newtheorem*{corollary*}{Следствие}
\newtheorem*{example}{Пример}
\newtheorem{observation}{Наблюдение}
\newtheorem*{prop}{Свойства}
\newtheorem*{obozn}{Обозначение}

% subtheorem
\makeatletter
\newenvironment{subtheorem}[1]{%
    \def\subtheoremcounter{#1}%
    \refstepcounter{#1}%
    \protected@edef\theparentnumber{\csname the#1\endcsname}%
    \setcounter{parentnumber}{\value{#1}}%
    \setcounter{#1}{0}%
    \expandafter\def\csname the#1\endcsname{\theparentnumber.\Alph{#1}}%
    \ignorespaces
}{%
    \setcounter{\subtheoremcounter}{\value{parentnumber}}%
    \ignorespacesafterend
}
\makeatother
\newcounter{parentnumber}

\newtheorem{manualtheoreminner}{Теорема}
\newenvironment{manualtheorem}[1]{%
    \renewcommand\themanualtheoreminner{#1}%
    \manualtheoreminner
}{\endmanualtheoreminner}

\newcommand{\dbltilde}[1]{\accentset{\approx}{#1}}
\newcommand{\intt}{\int\!}

% magical thing that fixes paragraphs
\makeatletter
\patchcmd{\CatchFBT@Fin@l}{\endlinechar\m@ne}{}
{}{\typeout{Unsuccessful patch!}}
\makeatother

\newcommand{\get}[2]{
    \ExecuteMetaData[#1]{#2}
}

\newcommand{\getproof}[2]{
    \iftoggle{useproofs}{\ExecuteMetaData[#1]{#2proof}}{}
}

\newcommand{\getwithproof}[2]{
    \get{#1}{#2}
    \getproof{#1}{#2}
}

\newcommand{\import}[3]{
    \subsection{#1}
    \getwithproof{#2}{#3}
}

\newcommand{\given}[1]{
    Дано выше. (\ref{#1}, стр. \pageref{#1})
}

\renewcommand{\ker}{\text{Ker }}
\newcommand{\im}{\text{Im }}
\renewcommand{\grad}{\text{grad}}
\newcommand{\rg}{\text{rg}}
\newcommand{\defeq}{\stackrel{\text{def}}{=}}
\newcommand{\defeqfor}[1]{\stackrel{\text{def } #1}{=}}
\newcommand{\itemfix}{\leavevmode\makeatletter\makeatother}
\newcommand{\?}{\textcolor{red}{???}}
\renewcommand{\emptyset}{\varnothing}
\newcommand{\longarrow}[1]{\xRightarrow[#1]{\qquad}}
\DeclareMathOperator*{\esup}{\text{ess sup}}
\newcommand\smallO{
    \mathchoice
    {{\scriptstyle\mathcal{O}}}% \displaystyle
    {{\scriptstyle\mathcal{O}}}% \textstyle
    {{\scriptscriptstyle\mathcal{O}}}% \scriptstyle
    {\scalebox{.6}{$\scriptscriptstyle\mathcal{O}$}}%\scriptscriptstyle
}
\renewcommand{\div}{\text{div}\ }
\newcommand{\rot}{\text{rot}\ }
\newcommand{\cov}{\text{cov}}

\makeatletter
\newcommand{\oplabel}[1]{\refstepcounter{equation}(\theequation\ltx@label{#1})}
\makeatother

\newcommand{\symref}[2]{\stackrel{\oplabel{#1}}{#2}}
\newcommand{\symrefeq}[1]{\symref{#1}{=}}

% xrightrightarrows
\makeatletter
\newcommand*{\relrelbarsep}{.386ex}
\newcommand*{\relrelbar}{%
    \mathrel{%
        \mathpalette\@relrelbar\relrelbarsep
    }%
}
\newcommand*{\@relrelbar}[2]{%
    \raise#2\hbox to 0pt{$\m@th#1\relbar$\hss}%
    \lower#2\hbox{$\m@th#1\relbar$}%
}
\providecommand*{\rightrightarrowsfill@}{%
    \arrowfill@\relrelbar\relrelbar\rightrightarrows
}
\providecommand*{\leftleftarrowsfill@}{%
    \arrowfill@\leftleftarrows\relrelbar\relrelbar
}
\providecommand*{\xrightrightarrows}[2][]{%
    \ext@arrow 0359\rightrightarrowsfill@{#1}{#2}%
}
\providecommand*{\xleftleftarrows}[2][]{%
    \ext@arrow 3095\leftleftarrowsfill@{#1}{#2}%
}

\allowdisplaybreaks

\newcommand{\unfinished}{\textcolor{red}{Не дописано}}

% Reproducible pdf builds 
\special{pdf:trailerid [
<00112233445566778899aabbccddeeff>
<00112233445566778899aabbccddeeff>
]}
%</preamble>


\begin{document}

\setcounter{section}{167}

\section{}
Используя теорему о рекурсии, докажите, что язык программ, которые останавливаются на пустом вводе, является неразрешимым. Является ли этот язык перечислимым?

\begin{minted}{python}
p(_):
    if r(p):
        while True:
            pass
    else:
        return
\end{minted}

Да, перечислим:
\begin{minted}[escapeinside=||,mathescape=true]{python}
for t in |$\N$|:
    for p in |$\overline{1, t}$|:
        if p(tl=t).halts():
            print(p)
\end{minted}

\section{}
Используя теорему о рекурсии, докажите, что язык программ, которые не останавливаются на пустом вводе, является неразрешимым. Является ли этот язык перечислимым?

\begin{minted}{python}
p(_):
    if r(p):
        return
    else:
        while True:
            pass
\end{minted}

Язык не перечислим, т.к. если есть если \(L \in RE\) и \(L \in coRE\), то \(L \in R\)

\section{}
Используя теорему о рекурсии, докажите, что язык программ, которые допускают бесконечное число слов, является неразрешимым.

\begin{minted}{python}
p(_):
    if r(p):
        return True
    else:
        return False
\end{minted}

\section{}
Используя теорему о рекурсии, докажите, что язык программ, которые допускают свой собственный исходный код, является неразрешимым.


\begin{minted}{python}
p(x):
    if x != p:
        return False
    if r(p):
        return False
    else:
        return True
\end{minted}

\section{}
Докажите, что существуют две различные программы $p$ и $q$, такие что программа $p$ печатает текст программы $q$, а программа $q$ печатает текст программы $p$.

\section{}
Докажите, что существует бесконечная последовательность различных программ $p_i$, такая что $p_1$ печатает пустую строку, а $p_i$ печатает текст программы $p_{i-1}$.

То же самое, но каждая строка не содержит в себе определение себя.

\section{}
Докажите, что существует бесконечная последовательность различных программ $p_i$, такая что $p_i$ печатает текст программы $p_{i+1}$.

То же самое, но в \(i\)-той строке мы определяем \(i + 1\)-ую.

\section{}
Докажите, что для любого конечного $n$ существует последовательность программ $p_1, p_2, \ldots, p_n$, что $p_i$ печатает текст $p_{i+1}$, а $p_n$ печатает текст $p_1$.

\begin{minted}{python}
quine(i):
    return f'''print(s1 = {s1}
    s2 = {s2}
    ...
    sn = {sn}
    print(s{i + 1 % n}))'''
s1 = quine(1)
s2 = quine(2)
...
sn = quine(n)
print(s1)
\end{minted}

\texttt{quine} --- макрос, а не часть исходного кода.

% Аналогично 172, но с \(n\) строк.

\section{}
Докажите, что язык программ, для которых не существует более короткой программы, которая на любом входе ведёт себя так же, является неразрешимым.

\section{}
Докажите, что язык программ, для которых не существует программы такой же длины, которая на любом входе ведёт себя так же, является либо конечным, либо неразрешимым.

\section{}
Busy Beaver. Функция $BB(n)$ возвращает длину максимальной строки, которую программа длины $n$ может вывести на пустом входе и завершиться. Докажите, что $BB$ является невычислимой.

\section{}
Докажите, что для любой всюду определенной вычислимой функции $f$ найдется значение $n$, для которого $BB(n) > f(n)$.

Соуфивается

\section{}
Докажите, что для любой всюду определенной вычислимой функции $f$ найдется бесконечно много значений $n$, для которых $BB(n) > f(n)$.

\section{}
Колмогоровская сложность. $K(s)$ это длина минимальной программы, которая на пустом входе выводит строку $s$ и завершается. Докажите, что $K$ является невычислимой.

\section{}
Пусть для любой строки $s$ выполнено $K(s) \ge f(s)$, где $f$ — всюду определенная вычислимая функция. Докажите, что найдется константа $C$, такая что $f(s) \le C$ для любой $s$.

\end{document}
