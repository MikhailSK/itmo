\documentclass[12pt, a4paper]{article}

%<*preamble>
% Math symbols
\usepackage{amsmath, amsthm, amsfonts, amssymb}
\usepackage{accents}
\usepackage{esvect}
\usepackage{mathrsfs}
\usepackage{mathtools}
\mathtoolsset{showonlyrefs}
\usepackage{cmll}
\usepackage{stmaryrd}
\usepackage{physics}
\usepackage[normalem]{ulem}
\usepackage{ebproof}
\usepackage{extarrows}

% Page layout
\usepackage{geometry, a4wide, parskip, fancyhdr}

% Font, encoding, russian support
\usepackage[russian]{babel}
\usepackage[sb]{libertine}
\usepackage{xltxtra}

% Listings
\usepackage{listings}
\lstset{basicstyle=\ttfamily,breaklines=true}
\setmonofont[Scale=MatchLowercase]{JetBrains Mono}

% Miscellaneous
\usepackage{array}
\usepackage{booktabs}\renewcommand{\arraystretch}{1.2}
\usepackage{calc}
\usepackage{caption}
\usepackage{subcaption}
\captionsetup{justification=centering,margin=2cm}
\usepackage{catchfilebetweentags}
\usepackage{enumitem}
\usepackage{etoolbox}
\usepackage{float}
\usepackage{lastpage}
\usepackage{minted}
\usepackage{svg}
\usepackage{wrapfig}
\usepackage{xcolor}
\usepackage[makeroom]{cancel}

\newcolumntype{L}{>{$}l<{$}}
    \newcolumntype{C}{>{$}c<{$}}
\newcolumntype{R}{>{$}r<{$}}

% Footnotes
\usepackage[hang]{footmisc}
\setlength{\footnotemargin}{2mm}
\makeatletter
\def\blfootnote{\gdef\@thefnmark{}\@footnotetext}
\makeatother

% References
\usepackage{hyperref}
\hypersetup{
    colorlinks,
    linkcolor={blue!80!black},
    citecolor={blue!80!black},
    urlcolor={blue!80!black},
}

% tikz
\usepackage{tikz}
\usepackage{tikz-cd}
\usetikzlibrary{arrows.meta}
\usetikzlibrary{decorations.pathmorphing}
\usetikzlibrary{calc}
\usetikzlibrary{patterns}
\usepackage{pgfplots}
\pgfplotsset{width=10cm,compat=1.9}
\newcommand\irregularcircle[2]{% radius, irregularity
    \pgfextra {\pgfmathsetmacro\len{(#1)+rand*(#2)}}
    +(0:\len pt)
    \foreach \a in {10,20,...,350}{
            \pgfextra {\pgfmathsetmacro\len{(#1)+rand*(#2)}}
            -- +(\a:\len pt)
        } -- cycle
}

\providetoggle{useproofs}
\settoggle{useproofs}{false}

\pagestyle{fancy}
\lhead{Лабораторная работа №2}
\lfoot{Михайлов Максим}
\rfoot{M3337}
\cfoot{}
\rhead{стр. \thepage\ из \pageref*{LastPage}}

\newcommand{\R}{\mathbb{R}}
\newcommand{\Q}{\mathbb{Q}}
\newcommand{\Z}{\mathbb{Z}}
\newcommand{\B}{\mathbb{B}}
\newcommand{\N}{\mathbb{N}}
\renewcommand{\Re}{\mathfrak{R}}
\renewcommand{\Im}{\mathfrak{I}}

\newcommand{\const}{\text{const}}
\newcommand{\cond}{\text{cond}}

\newcommand{\teormin}{\textcolor{red}{!}\ }

\DeclareMathOperator*{\xor}{\oplus}
\DeclareMathOperator*{\equ}{\sim}
\DeclareMathOperator{\sign}{\text{sign}}
\DeclareMathOperator{\Sym}{\text{Sym}}
\DeclareMathOperator{\Asym}{\text{Asym}}

\DeclarePairedDelimiter{\ceil}{\lceil}{\rceil}

% godel
\newbox\gnBoxA
\newdimen\gnCornerHgt
\setbox\gnBoxA=\hbox{$\ulcorner$}
\global\gnCornerHgt=\ht\gnBoxA
\newdimen\gnArgHgt
\def\godel #1{%
    \setbox\gnBoxA=\hbox{$#1$}%
    \gnArgHgt=\ht\gnBoxA%
    \ifnum     \gnArgHgt<\gnCornerHgt \gnArgHgt=0pt%
    \else \advance \gnArgHgt by -\gnCornerHgt%
    \fi \raise\gnArgHgt\hbox{$\ulcorner$} \box\gnBoxA %
    \raise\gnArgHgt\hbox{$\urcorner$}}

% \theoremstyle{plain}

\theoremstyle{definition}
\newtheorem{theorem}{Теорема}
\newtheorem*{definition}{Определение}
\newtheorem{axiom}{Аксиома}
\newtheorem*{axiom*}{Аксиома}
\newtheorem{lemma}{Лемма}
\newenvironment{solution}[1][Решение.]{\begin{proof}[#1]}{\end{proof}}

\theoremstyle{remark}
\newtheorem*{remark}{Примечание}
\newtheorem*{exercise}{Упражнение}
\newtheorem{corollary}{Следствие}[theorem]
\newtheorem*{statement}{Утверждение}
\newtheorem*{corollary*}{Следствие}
\newtheorem*{example}{Пример}
\newtheorem{observation}{Наблюдение}
\newtheorem*{prop}{Свойства}
\newtheorem*{obozn}{Обозначение}

% subtheorem
\makeatletter
\newenvironment{subtheorem}[1]{%
    \def\subtheoremcounter{#1}%
    \refstepcounter{#1}%
    \protected@edef\theparentnumber{\csname the#1\endcsname}%
    \setcounter{parentnumber}{\value{#1}}%
    \setcounter{#1}{0}%
    \expandafter\def\csname the#1\endcsname{\theparentnumber.\Alph{#1}}%
    \ignorespaces
}{%
    \setcounter{\subtheoremcounter}{\value{parentnumber}}%
    \ignorespacesafterend
}
\makeatother
\newcounter{parentnumber}

\newtheorem{manualtheoreminner}{Теорема}
\newenvironment{manualtheorem}[1]{%
    \renewcommand\themanualtheoreminner{#1}%
    \manualtheoreminner
}{\endmanualtheoreminner}

\newcommand{\dbltilde}[1]{\accentset{\approx}{#1}}
\newcommand{\intt}{\int\!}

% magical thing that fixes paragraphs
\makeatletter
\patchcmd{\CatchFBT@Fin@l}{\endlinechar\m@ne}{}
{}{\typeout{Unsuccessful patch!}}
\makeatother

\newcommand{\get}[2]{
    \ExecuteMetaData[#1]{#2}
}

\newcommand{\getproof}[2]{
    \iftoggle{useproofs}{\ExecuteMetaData[#1]{#2proof}}{}
}

\newcommand{\getwithproof}[2]{
    \get{#1}{#2}
    \getproof{#1}{#2}
}

\newcommand{\import}[3]{
    \subsection{#1}
    \getwithproof{#2}{#3}
}

\newcommand{\given}[1]{
    Дано выше. (\ref{#1}, стр. \pageref{#1})
}

\renewcommand{\ker}{\text{Ker }}
\newcommand{\im}{\text{Im }}
\renewcommand{\grad}{\text{grad}}
\newcommand{\rg}{\text{rg}}
\newcommand{\defeq}{\stackrel{\text{def}}{=}}
\newcommand{\defeqfor}[1]{\stackrel{\text{def } #1}{=}}
\newcommand{\itemfix}{\leavevmode\makeatletter\makeatother}
\newcommand{\?}{\textcolor{red}{???}}
\renewcommand{\emptyset}{\varnothing}
\newcommand{\longarrow}[1]{\xRightarrow[#1]{\qquad}}
\DeclareMathOperator*{\esup}{\text{ess sup}}
\newcommand\smallO{
    \mathchoice
    {{\scriptstyle\mathcal{O}}}% \displaystyle
    {{\scriptstyle\mathcal{O}}}% \textstyle
    {{\scriptscriptstyle\mathcal{O}}}% \scriptstyle
    {\scalebox{.6}{$\scriptscriptstyle\mathcal{O}$}}%\scriptscriptstyle
}
\renewcommand{\div}{\text{div}\ }
\newcommand{\rot}{\text{rot}\ }
\newcommand{\cov}{\text{cov}}

\makeatletter
\newcommand{\oplabel}[1]{\refstepcounter{equation}(\theequation\ltx@label{#1})}
\makeatother

\newcommand{\symref}[2]{\stackrel{\oplabel{#1}}{#2}}
\newcommand{\symrefeq}[1]{\symref{#1}{=}}

% xrightrightarrows
\makeatletter
\newcommand*{\relrelbarsep}{.386ex}
\newcommand*{\relrelbar}{%
    \mathrel{%
        \mathpalette\@relrelbar\relrelbarsep
    }%
}
\newcommand*{\@relrelbar}[2]{%
    \raise#2\hbox to 0pt{$\m@th#1\relbar$\hss}%
    \lower#2\hbox{$\m@th#1\relbar$}%
}
\providecommand*{\rightrightarrowsfill@}{%
    \arrowfill@\relrelbar\relrelbar\rightrightarrows
}
\providecommand*{\leftleftarrowsfill@}{%
    \arrowfill@\leftleftarrows\relrelbar\relrelbar
}
\providecommand*{\xrightrightarrows}[2][]{%
    \ext@arrow 0359\rightrightarrowsfill@{#1}{#2}%
}
\providecommand*{\xleftleftarrows}[2][]{%
    \ext@arrow 3095\leftleftarrowsfill@{#1}{#2}%
}

\allowdisplaybreaks

\newcommand{\unfinished}{\textcolor{red}{Не дописано}}

% Reproducible pdf builds 
\special{pdf:trailerid [
<00112233445566778899aabbccddeeff>
<00112233445566778899aabbccddeeff>
]}
%</preamble>


\begin{document}

\setcounter{section}{147}

\section{}
Докажите, что если $A$ неперечислимо и $A \le_m B$, то $B$ неперечислимо.

Пусть \(B\) перечислимо. Тогда \(B\) полуразрешим и есть полуразрешитель \(p\). \(p \circ f\) есть полуразрешитель для \(A\) --- противоречие.

\section{}
Пусть $A$ перечислимо и $\mathbb{N} \setminus A \le_m A$. Что можно сказать про $A$?

\(A\) перечислимо \(\Rightarrow \mathbb{N} \setminus A\) перечислимо \(\Rightarrow A\) разрешимо.

\section{}
Пусть $A$ перечислимо и $A \le_m \mathbb{N} \setminus A$. Что можно сказать про $A$?

Кажется, ничего, т.к. его дополнение может быть не перечислимым, а следовательно он может быть не разрешимым.

\section{}
Пусть дана функция $f : A \to \mathbb{N}$. Ее продолжением на множество $B \supset A$ называется функция $g:B \to \mathbb{N}$, что если $x\in A$, то $g(x) = f(x)$. Докажите, что существует вычислимая функция $f$, у которой не существует всюду определенного вычислимого продолжения.

% \subsection{Вычислимые функции можно занумеровать}

\(\exists \mathcal{F} : \N \to \N \to \N : \forall\) вычислимой \(f\) \(\exists k \in \N : f(\circ) = \mathcal{F}(k, \circ)\) --- очевидно, т.к. можно рассмотреть какую-либо модель вычислений \textit{(машины Тьюринга или ваш любимый язык программирования)}, найти для \(f\) программу \(p\) и закодировать её как число \(n \in \N\). Несложно заметить, что \(\mathcal{F}\) вычислима, но не тотальна, т.е. \(\mathcal{F} : \mathfrak{N} \subset \N \to \N \to \N\).

Искомая функция \(f : \mathfrak{N} \to \N \to \N : n \mapsto \mathcal{F}(n, n) + 42\). Почему она не продолжима до вычислимой? Пусть её продолжает \(g\). Ей соответствует некоторый индекс \(k_g\), такой что \(\mathcal{F}(k_g, n) = g(n)\). В силу этого \(\mathcal{F}\) определена на \((k_g, n) \ \ \forall n\). Но на \(\mathfrak{N}\) \(g\) есть \(f\), т.е. \(\mathcal{F}(k_g, n) = f(n) = \mathcal{F}(n, n) + 42\). Подставим \(n = k_g\), тогда \(\mathcal{F}(n, n) = \mathcal{F}(n, n) + 42\) --- противоречие.

\section{}
Два перечислимых множества $A$ и $B$, где $A \cap B = \emptyset$ называются неотделимыми, если не существует разрешимых множеств $X$ и $Y$, таких что $A \subset X$, $B \subset Y$, $X \cap Y = \emptyset$. Покажите, что существуют неотделимые множества. Указание: рассмотрите множества пар $\langle p, x\rangle$, где $p$ - программа, возвращающая целое число, для некоторого условия.

Рассмотрим функцию как из предыдущего номера, но \(n \mapsto !\mathcal{F}(n, n)\). По аналогичному утверждению \(f\) не имеет вычислимого продолжения. Тогда пусть \(A = \{n\ |\ f(n) = 0\}\) и \(B = \{n\ |\ f(n) = 1\}\). Предположим, что \(\exists X, Y\). Но характеристическая функция \(X\) есть вычислимое продолжение \(f\) --- противоречие.

\section{}
Обобщите определение неотделимых множеств на счетное семейство множеств. Докажите, что существует счетное семейство неотделимых множеств.

То же самое, но не делать \(!\), т.е. \(A_i = \{n\ |\ f(n) = i\}, f : n \mapsto \mathcal{F}(n, n) + 1\)

\(\{A_i\}_{i = 1}^{+\infty}\) --- неотделимы, если \(\forall i \ \ A_i\) перечислимо \(\exists ! \{B_i\} : \forall i \ \ A_i \subset B_i, B_i\) разрешимо \(\bigcap B_i = \emptyset\)

\section{}
Докажите, что множество программ, допускающих заданное конечное множество слов $x_1, \ldots, x_n$, перечислимо, но не разрешимо.

\subsection{Перечислимо}
\begin{minted}[escapeinside=||,mathescape=true]{python}
for t in |$\N$|:
    for prog in |$\N/_t$|:
        for x in xs:
            if prog(x, tl=t) == 1:
                print(prog)
\end{minted}

Если подразумевается, что ещё не допускаются никакие другие слова, то нужно добавить:
\begin{minted}[escapeinside=||,mathescape=true]{python}
        for w in |$\left(\sum^* \setminus xs\right)/_t$|:
            if prog(x, tl=t) != 1:
                print(prog)
\end{minted}

\subsection{Не разрешимо}

Пусть есть разрешитель \(P\). Тогда построим следующую программу:
\begin{minted}[escapeinside=||,mathescape=true]{cpp}
f(x):
    return P(f) ^ |$x \in \{x_i\}$|
\end{minted}

По классическому аргументу Тьюринга получаем противоречие.

\section{}
Докажите, что множество программ, допускающих бесконечное множество слов не разрешимо.

Пусть есть разрешитель \(P\). Тогда построим следующую программу:
\begin{minted}[escapeinside=||,mathescape=true]{c}
f(x):
    return !P(f)
\end{minted}

Если \(P(f) = 1\), то \(f \equiv 0\), а следовательно не допускает никакие слова, следовательно не допускает бесконечное множество слов. Если \(P(f) = 0\), то \(f\) допускает все слова, а их бесконечно.

\section{}
Докажите, что множество программ, зависающих на любом входе, не разрешимо.

Пусть есть разрешитель \(P\). Тогда построим следующую программу:
\begin{minted}[escapeinside=||,mathescape=true]{c}
f(x):
    if P(f)
        return 1;
    else
        while (true);
\end{minted}

Противоречие.

\section{}
Докажите, что множество программ, останавливающихся на своём собственном исходном коде, перечислимо, но не разрешимо.

\subsection{Перечислимо}
\begin{minted}[escapeinside=||,mathescape=true]{python}
for t in |$\N$|:
    for prog in |$\N/_t$|:
         if prog(prog, tl=t) != |$\perp$|:
            print(prog)
\end{minted}

\subsection{Не разрешимо}

\begin{minted}[escapeinside=||,mathescape=true]{c}
f(x):
    if P(f)
        return 1;
    else
        while (true);
    \end{minted}

\section{}
Покажите, что следующие три свойства множества $X$ равносильны: (1) $X$ можно представить в виде $A \setminus B,$ где $A$ — перечислимое множество, а $B$ — его перечислимое подмножество; (2) $X$ можно представить в виде $A \setminus B$, где $A$ и $B$ — перечислимые множества; (3) $X$ можно представить в виде симметрической разности двух перечислимых множеств.

\section{}
Покажите, что множество $X$ можно представить в виде $A\setminus (B \setminus C)$, где $A \supset B \supset C$ — перечислимые множества, если и только если его можно представить в виде симметрической разности трёх перечислимых множеств.

\section{}
Покажите, что существует множество, которое можно представить в виде симметрической разности трёх перечислимых множеств, но нельзя представить в виде симметрической разности двух перечислимых множеств

\section{}
Язык ограниченной задачи останова (bounded halting) $BH = \{ (p, t) | p$ завершается на пустом входе за $t$ шагов $\}$. Докажите, что $BH$ разрешим.

\begin{minted}{python}
D(p, t):
    return p(tl=t).halts()
\end{minted}

\section{}
Докажите, что существует разрешимое множество пар, проекция которого на одну из осей не является разрешимой.

Это \(BH\), т.к. проекция на первую ость есть \(HALT\) --- программы, которые завершаются на пустом входе за произвольное конечное число шагов.

\section{}
Докажите, что существует разрешимое множество пар, проекция которого на каждую из осей не является разрешимой.

% Как в предыдущей задаче, но вместо \(t\) будут программы, возвращающие \(t\) на пустом входе. Проекция на первую ось всё ещё \(HALT\), а на вторую тоже \(HALT\).

\section{}
Некоторое множество $S$ натуральных чисел разрешимо. Разложим все числа из $S$ на простые множители и составим множество $D$ всех простых чисел, встречающихся в этих разложениях. Можно ли утверждать, что множество $D$ перечислимо?

\begin{minted}{python}
listD():
    for s in listS():
        print(s.factorize())
\end{minted}

\section{}
Некоторое множество $S$ натуральных чисел разрешимо. Разложим все числа из $S$ на простые множители и составим множество $D$ всех простых чисел, встречающихся в этих разложениях. Можно ли утверждать, что множество $D$ разрешимо?

Нет. Пусть \(p_k\) --- \(k\)-тое простое число. Пусть \(a_i\) есть произведение всех \(p_k\) таких, что \(k < i\) и \(k\), интерпретированное как программа, на пустом вводе останавливается за \(i\) шагов. Очевидно функция \(i \mapsto a_i\) вычислима по определению. Кроме того, она возрастающая, следовательно её образ разрешим. Пусть \(S\) есть образ \(i \mapsto a_i\). Определить, лежит ли число \(p_d \in D\) эквивалентно \(HALT\), т.к. \(p_d \in D \Leftrightarrow \exists i : d.halts(tl = i) \Leftrightarrow d.halts()\)

\section{}
Множество $A \subset \mathbb{N} \times \mathbb{N}$ разрешимо. Можно ли утверждать, что множество «нижних точек» множества $A$, то есть множество $B = \{\langle x,y\rangle | (\langle x,y\rangle \in A)$ и $(\langle x,z\rangle \not\in A$ для всех $z < y)\}$ является разрешимым?

\begin{minted}{python}
B(x, y):
    if not A(x, y):
        return False
    for z in 0..y-1:
        if A(x, z):
            return False
    return True
\end{minted}

Всегда останавливается, т.к. \(A(x, y)\) всегда останавливается.

\section{}
В предыдущем задании можно ли утверждать, что $B$ перечислимо, если $A$ перечислимо?


\end{document}