\input{preamble.sty}

\begin{document}

\setcounter{section}{184}

\section{}
Вещественное число $\alpha$ называется вычислимым, если существует вычислимая функция $a$, которая по любому рациональному $\varepsilon > 0$ даёт рациональное приближение к $\alpha$ с ошибкой не более $\varepsilon$, то есть $|\alpha − a(\varepsilon)| \le \varepsilon$ для любого рационального $\varepsilon > 0$. Докажите, что число $\alpha$ вычислимо тогда и только тогда, когда множество рациональных чисел, меньших $\alpha$, разрешимо.

\begin{itemize}
    \item [\(\Rightarrow\)]
          \begin{minted}{kotlin}
decider(b):
    for eps in 1..inf.map { exp(-it) }:
        if a(eps) != b.cutoff(eps):
            return a(eps) > b
          \end{minted}
    \item [\(\Leftarrow\)] Бинпоиск
\end{itemize}

\section{}
Докажите, что число $\alpha$ вычислимо тогда и только тогда, когда последовательность знаков представляющей его десятичной (или двоичной) дроби вычислима. Последовательность называется вычислимой, если существует программа, которая по номеру $i$ выдает соответствующий элемент последовательности $a_i$.

\begin{itemize}
    \item [\( \Rightarrow \)] \(\varepsilon 10^{ - i}\)
    \item [\( \Leftarrow \)] аналогично
\end{itemize}

\section{}
Докажите, что число $\alpha$ вычислимо тогда и только тогда, когда существует вычислимая последовательность рациональных чисел, вычислимо сходящаяся к $\alpha$ (последнее означает, что можно алгоритмически указать $N$ по $\varepsilon$ в стандартном $\varepsilon$-$N$-определении сходимости.)

Ну очевидно же

\section{}
Покажите, что сумма, произведение, разность и частное вычислимых вещественных чисел вычислимы.

Можно перейти к подпредельным последовательностям.

\section{}
Покажите, что корень многочлена с вычислимыми коэффициентами вычислим.

Запустим алгоритм нахождения корней полинома.

\section{}
Сформулируйте и докажите утверждение о том, что предел вычислимо сходящейся последовательности вычислимых вещественных чисел вычислим.

Дана последовательность \(\{a_i\}\), пусть её предел \(a\). Покажем, что \(a \in \mathcal{E}\).
\begin{minted}{python}
a(eps):
    return a_i(N(eps))
\end{minted}

\section{}
Вещественное число $\alpha$ называют перечислимым снизу, если множество всех рациональных чисел, меньших $\alpha$, перечислимо. (Перечислимость сверху определяется аналогично.) Докажите, что число $\alpha$ перечислимо снизу тогда и только тогда, когда оно является пределом некоторой вычислимой возрастающей последовательности рациональных чисел.

\href{https://neerc.ifmo.ru/wiki/index.php?title=%D0%92%D1%8B%D1%87%D0%B8%D1%81%D0%BB%D0%B8%D0%BC%D1%8B%D0%B5_%D1%87%D0%B8%D1%81%D0%BB%D0%B0}{Cоуфивается} 

\section{}
Докажите, что действительное число вычислимо тогда и только тогда, когда оно перечислимо снизу и сверху.

\href{https://neerc.ifmo.ru/wiki/index.php?title=%D0%92%D1%8B%D1%87%D0%B8%D1%81%D0%BB%D0%B8%D0%BC%D1%8B%D0%B5_%D1%87%D0%B8%D1%81%D0%BB%D0%B0}{Cоуфивается} 

\section{}
Докажите, что множество функций-приближений для рациональных вычислимых чисел $\alpha$ является неразрешимым. Указание: вспомните теорему о рекурсии.

Пусть дан \texttt{decider}.

\begin{minted}{python}
f(eps):
    if decider(f):
        return 1 / eps
    else:
        return 0
\end{minted}

Если \texttt{decider(f)}, то \texttt{f} не выдает аппроксимацию никакого числа. Иначе \texttt{f} выдает аппроксимацию 0.

\section{}
Покажите, что существуют перечислимые снизу, но не вычислимые числа. Указание: рассмотрим сумму ряда $\sum 2^{-k}$ по $k$ из какого-либо множества $P$.

\href{https://neerc.ifmo.ru/wiki/index.php?title=%D0%92%D1%8B%D1%87%D0%B8%D1%81%D0%BB%D0%B8%D0%BC%D1%8B%D0%B5_%D1%87%D0%B8%D1%81%D0%BB%D0%B0}{Cоуфивается}

\section{}
Приведите пример невычислимого предела сходящейся (но не вычислимо) последовательности вычислимых чисел

Последовательность --- частичные суммы из 194.

\section{}
Приведите пример невычислимого предела вычислимо сходящейся (но не вычислимой) последовательности вычислимых чисел

195 но для \(2^{ -BB(i)}\). Невычислимость предела очевидна, как и вычислимость произвольного элемента последовательности. Т.к. функция Аккермана вычислима, \(BB\) растёт быстрее оной, т.е. \(BB(i) > A(i, i)\). Там возникает небольшая несостыковка для начальных значений, проще сказать \(BB(2i + 1) > A(i - 1, i - 1)\). Тогда \(a_{2i + 1} < 2^{ -A(i - 1, i - 1)}\), а следовательно \(\sum_{j = 2i + 1}^{+\infty} a_{2i + 1} < 2 \cdot 2^{ -A(i - 1, i - 1)}\) и тогда для \(\varepsilon\) можно выдать \(N = \min \{i : 2 \cdot  2^{ -A(i - 1, i - 1)}\} \), что вычислимо за конечное время.

\section{}
Множество $A$ называется эффективно бесконечным, если существует всюду определенная вычислимая функция $f$, которая по числу $n$ выводит $n$ различных элементов множества $A$. Докажите, что если множество $A$ содержит бесконечное перечислимое подмножество, то оно эффективно бесконечно.

\begin{minted}{python}
f(n):
    enumerator().limit(n)
\end{minted}

\section{}
Докажите, что если множество $A$ эффективно бесконечно, то оно содержит бесконечное перечислимое подмножество.

\begin{minted}{python}
enumerator():
    for i in 1..n:
        f(i)
\end{minted}

\section{}
Обозначим как $L(p)$ множество слов, которые допускается программой $p$. Множество $A$ называется эффективно неперечислимым, если существует всюду определенная вычислимая функция $f$, которая по программе $p$ указывает слово $x$, такое что $x \in L(p) \oplus A$. Докажите, что дополнение к диагонали универсального множества $\overline D$, где $D = \left\{p | \langle p, p\rangle \in U\right\}$, является эффективно неперечислимым.

\(f(p)\) выдает слово \(x\), такое что либо \(p(x) = 1\) и при этом \(x \in D \Leftrightarrow x(x) = 1\), либо ни одно из этого. \(f(p) = p\).

\section{}
Докажите, что дополнение к универсальному множеству $\overline U$ является эффективно неперечислимым.

\(f(p)\) выдает слово \(\ev{a, b}\), такое что либо \(p(\ev{a, b}) = 1\) и при этом \(\ev{a, b} \in U \Leftrightarrow a(b) = 1\), либо ни одно из этого.
\begin{minted}{python}
a(b):
    return p((a, b))

f(p):
    return (a, b)
\end{minted}

\section{}
Докажите, что любое эффективно неперечислимое множество является эффективно бесконечным.

\begin{minted}{python}
gen(n):
    xs = {}
    for i in 1..n:
        xs += f(xs::contains)
    print xs
\end{minted}

Сначала xs пустой, следовательно \(L(p) = \emptyset \Rightarrow f\) выдает элемент \(A\). На следующей итерации \(f\) не может выдать тот же самый элемент, т.к. он \(\in L(p)\). Кроме того, \(L(p) \subset A\), следовательно \(f\) не выдаст элемента \(L(p)\). Остается выдать новый элемент \(A\).

\section{}
Множество называется иммунным, если оно бесконечно, но не содержит бесконечных перечислимых подмножеств. Перечислимое множество называется простым, если дополнение к нему иммунно. Докажите, что существует простое множество.

Простое множество --- множество, которое пересекается со всеми бесконечными перечислимыми множествами и дополнение которого конечно.

Рассмотрим \(\{\ev{x, y}\ |\ y \in W_x, y > 2x\}\), где \(W_x\) есть образ \(x\) как программы. Множество, очевидно, перечислимо. Упорядочим его каким-либо образом и возьмём для каждого \(x\) его первое вхождение вида \(\ev{x, y}\), сохранив все такие \(y\) в множество \(\mathfrak{C}\). Перечислимость сохраняется. По построению \(\forall k\) из множества \(\overline{1, 2k}\) в \(\mathfrak{C}\) входит не больше чем \(k\) чисел, следовательно множество c дырками и его дополнение бесконечно. Докажем условие про пересечение.

Рассмотрим бесконечное перечислимое множество \(A\). Пусть оно перечисляется некоторым \(x\). В силу бесконечности \(A\) в нём есть числа \( > 2x\), кроме того в \(\mathfrak{C}\) тоже есть такие числа по построению. Таким образом, в \(C\) лежало \(\ev{x, y}\) для \(y > 2x\) и минимальное такое \(y\) лежит в \(\mathfrak{C}\), следовательно \(A \cap \mathfrak{C} \neq  \emptyset\).

\section{}
Докажите, что множество является иммунным тогда и только тогда, когда оно не содержит бесконечных разрешимых подмножеств.

В одну сторону очевидно, т.к. определение выполнено.



\end{document}
