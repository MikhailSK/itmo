\input{preamble.sty}

\begin{document}

\setcounter{section}{81}

\section{}
Будем обозначать \(Seq_T\), \(Cyc_T\), \(Set_T\) соответственно последовательности, циклы и множества, размер которых принадлежит множеству \(T\). Опишите класс помеченных объектов \(Set(Cyc_{> 1}(Z))\). Найдите его экспоненциальную производящую функцию.'

\(Set(Cyc_{ > 1}(Z))\) есть множество перестановок без \(i\mapsto i\).

\begin{align*}
    Cyc_{ > 1}(Z)     & = \sum_{i = 2}^{+\infty} Cyc_i(Z)      \\
                      & = \sum_{i = 2}^{+\infty} \frac{t^i}{i} \\
                      & = - t - \ln(1 - t)                     \\
    Set(Cyc_{> 1}(Z)) & = \exp(- t - \ln(1 - t))               \\
                      & = \frac{1}{(1 - t)e^{t}}
\end{align*}

\section{}
Для производящей функции из прошлого задания найдите явную формулу и асимптотическое поведение количества объектов веса \(n\).

\begin{align*}
    Set(Cyc_{> 1}(Z))     & = \frac{1}{(1 - t)e^{t}}                              \\
    [n]Set(Cyc_{ > 1}(Z)) & = \sum_{k = 0}^n \binom{n}{k} k! \cdot ( - 1)^{n - k} \\
                          & = \sum_{k = 0}^n ( - 1)^{n - k} \frac{n!}{(n - k)!}   \\
                          & = n! \sum_{k = 0}^n \frac{( - 1)^{n - k}}{(n - k)!}   \\
                          & \to n! e^{ - 1}                                       \\
\end{align*}

\section{}
Опишите класс помеченных объектов \(Set(Cyc_{1, 2}(Z))\). Найдите его экспоненциальную производящую функцию.

Это множество инволюций.

\begin{align*}
    Set(Cyc_{1, 2}(Z)) & = \exp\left( t + \frac{t^2}{2} \right)
\end{align*}

\section{}
Сюрьекции на \(r\)-элементное множество. Осознайте, что \(Seq_{=r}(Set_{\ge 1}(Z))\) задаёт сюрьекции на \(r\)-элементное множество. Найдите экспоненциальную производящую функцию.

Для \(i\)-го элемента множества мы берем \(i\)-тый член последовательности, который является непустым множеством, т.е. \(\forall i \ \ f^{ - 1}(a_i) \neq \emptyset\), что есть определение сюрьекции.

\begin{align*}
    Seq_{ = r}(Set_{\ge 1}(Z)) & = \left( \sum_{i = 1}^{+\infty} \frac{t^i}{i!} \right)^r \\
                               & = (e^t - 1)^r
\end{align*}

\section{}
Разбиения на \(r\) множеств. Осознайте, что \(Set_{=r}(Set_{\ge 1}(Z))\) задаёт разбиения на \(r\) множеств. Найдите экспоненциальную производящую функцию. Что стоит при \(z^n\)?

Очевидно, т.к. множество из \(r\) элементов, каждый из которых --- непустое множество и есть разбиение.

\begin{align*}
    Set_{=r}(Set_{\ge 1}(Z)) & = \frac{(e^t - 1)^r}{r!}
\end{align*}

\begin{align*}
    [n](e^t - 1)^r            & = [n]\left( \sum_{i = 0}^{+\infty} \frac{t^i}{i!} - 1 \right)^r                                         \\
                              & = [n] \sum_{k = 0}^{r} \binom{r}{k} \left(\sum_{i = 0}^{+\infty} \frac{t^i}{i!}\right)^k ( - 1)^{n - k} \\
                              & = \sum_{k = 0}^{r} \binom{r}{k} \frac{(r - i)^{n}}{n!} ( - 1)^{n - k}                                   \\
    [n]\frac{(e^t - 1)^r}{r!} & = \sum_{k = 0}^{r} \binom{r}{k} \frac{(r - i)^{n}}{n!r!} ( - 1)^{n - k}                                 \\
\end{align*}

\section{}
Числа Белла. Число Белла \(b_n\) равно числу разбиений \(n\)-элементного множества на подмножества (число подмножеств не фиксировано). Докажите, что экспоненциальная производящая функция для чисел Белла равна \(e^{e^z-1}\).

\begin{align*}
    \sum_{r = 0}^{+\infty} \frac{(e^t - 1)^r}{r!} & = e^{e^t - 1}
\end{align*}

\section{}
Гиперболический синус \(\mathrm{sh}\,z\) равен \(\frac{1}{2}(e^{z}-e^{-z})\). Гиперболический косинус \(\mathrm{ch}\,z\) равен \(\frac{1}{2}(e^{z}+e^{-z})\). Рассмотрим разбиения \(n\)-элементного множества на непустые подмножества. Докажите, что для разбиений на нечетное число подмножеств экспоненциальная производящая функция равна \(\mathrm{sh}(e^z-1)\).

\begin{align*}
    \sum_{r = 0}^{+\infty} \frac{(e^t - 1)^{2r + 1}}{(2r + 1)!} & = \frac{e^{e^t - 1} - e^{1 - e^t}}{2} \\
                                                                & = \mathrm{sh}(e^z-1)
\end{align*}

\section{}
Докажите, что для разбиений на четное число подмножеств экспоненциальная производящая функция равна \(\mathrm{ch}(e^z-1)\).

\begin{align*}
    \sum_{r = 0}^{+\infty} \frac{(e^t - 1)^{2r}}{(2r)!} & = \frac{e^{e^t - 1} + e^{1 - e^t}}{2} \\
                                                        & = \mathrm{ch}(e^z-1)
\end{align*}

\section{}
Докажите, что для разбиений на произвольное число подмножеств, каждое из которых содержит нечетное число элементов, экспоненциальная производящая функция равна \(e^{\mathrm{sh}\,z}\).


\begin{align*}
    Set(Set_{\equiv 1\bmod 2}(Z)) & = \exp\left( \sum_{i = 0}^{+\infty} \frac{t^{2i + 1}}{(2i + 1)!} \right) \\
                                  & = \exp\sh t                                                              \\
\end{align*}

\section{}
Докажите, что для разбиений на произвольное число подмножеств, каждое из которых содержит четное число элементов, экспоненциальная производящая функция равна \(e^{\mathrm{ch}\,z-1}\). Почему здесь в показателе степени есть \(-1\), а в предыдущем задании нет?

\begin{align*}
    Set(Set_{\equiv 0\bmod 2}(Z)) & = \exp\left( \sum_{i = 1}^{+\infty} \frac{t^{2i}}{(2i)!} \right) \\
                                  & = \exp(\ch t - 1)
\end{align*}

Потому что нужно забанить пустые множества.

\section{}
Обобщите четыре предыдущих задания. Как выглядят экспоненциальные производящие функции для разбиений на (не)четное число подмножеств, каждое из которых содержит (не)четное число элементов? (Необходимо дать четыре ответа для всех комбинаций)

Тривиально.

\begin{tabular}{p{5cm}|c|c}
    \toprule
    Здесь могла быть ваша реклама & Чётное число подмножеств & Нечётное число подмножеств \\
    \midrule
    Чётное число элементов        & \(\ch (\ch z - 1)\)      & \(\sh (\ch z - 1)\)        \\
    Нечётное число элементов      & \(\ch (\sh z)\)          & \(\sh (\sh z)\)            \\
    \bottomrule
\end{tabular}

\section{}
Постройте экспоненциальную производящую функцию для перестановок, состоящих из четных циклов.

\begin{align*}
    Set(Cyc_{\equiv 0\bmod 2}) & = \exp\left( \sum_{i = 0}^{+\infty} \frac{t^{2i}}{2i} \right)            \\
                               & = \exp\left( \frac{1}{2}\sum_{i = 0}^{+\infty} \frac{t^{2i}}{i} \right)  \\
                               & = \exp\left( \frac{1}{2}\sum_{i = 0}^{+\infty} \frac{(t^2)^i}{i} \right) \\
                               & = \exp\left( \frac{1}{2} \ln \frac{1}{1 - t^2} \right)                   \\
                               & = \frac{1}{\sqrt{1 - t^2}}                                               \\
\end{align*}

\section{}
Постройте экспоненциальную производящую функцию для перестановок, состоящих из нечетных циклов.

\begin{align*}
    Set(Cyc_{\equiv 0\bmod 2})
     & = \exp\left( \sum_{i = 0}^{+\infty} \frac{t^{2i + 1}}{2i + 1} \right)                                \\
     & = \exp\left( \sum_{i = 1}^{+\infty} \frac{t^i}{i} - \sum_{i = 1}^{+\infty} \frac{t^{2i}}{2i} \right) \\
     & = \exp\left( \ln \frac{1}{1 - t} - \frac{1}{2} \ln \frac{1}{1 - t^2} \right)                         \\
     & = \exp \frac{1}{2}\left( \ln \frac{1}{(1 - t)^2} + \ln (1 - t^2) \right)                             \\
     & = \exp \frac{1}{2} \ln \frac{1 + t}{1 - t}                                                           \\
     & = \sqrt{\frac{1 + t}{1 - t}}                                                                         \\
\end{align*}

\section{}
Докажите, что для четного \(n\) количество перестановок, в которых все циклы четные, и количество перестановок, в которых все циклы нечетные, совпадают.

% Из матана\footnote{вольфрама} мы знаем, что:
% \begin{align*}
%     \frac{1}{\sqrt{1 - t^2}}
%      & = \sum_{n = 0}^{+\infty} ( - 1)^n \binom{ - \frac{1}{2}}{n} t^{2n}      \\
%      & = \sum_{n = 0}^{+\infty} ( - 1)^{n + 1}\binom{\frac{1}{2}}{n} t^{2n}    \\
%      & = \sum_{n = 0}^{+\infty} \binom{2n}{n} \frac{1}{2^{2n}(2n - 1)}  t^{2n} \\
% \end{align*}

% \begin{align*}
%     \sqrt{1 + t}
%      & = \sum_{n = 0}^{+\infty} \binom{\frac{1}{2}}{n} t^{n}                            \\
%      & = \sum_{n = 0}^{+\infty} \binom{2n}{n}\frac{( - 1)^{n + 1}}{2^{2n}(2n - 1)}t^{n} \\
% \end{align*}

% \begin{align*}
%     \frac{1}{\sqrt{1 - t}}
%      & = \sum_{n = 0}^{+\infty} \binom{2n}{n} \frac{1}{2^{2n}(2n - 1)} t^{n} \\
% \end{align*}

% \begin{align*}
%     [2n]\sqrt{\frac{1 + t}{1 - t}}
%      & = \sum_{k = 0}^{2n} \binom{2k}{k} \frac{( - 1)^{n + 1}}{2^{2k}(2k - 1)} \binom{2(2n - k)}{2n - k} \frac{1}{2^{2(2n - k)}(2(2n - k) - 1)} \\
%      & = \frac{1}{2^{2n}} \sum_{k = 0}^{2n} \binom{2k}{k} \frac{( - 1)^{n + 1}}{(2k - 1)} \binom{4n - 2k}{2n - k} \frac{1}{(4n - 2k - 1)}       \\
% \end{align*}

\begin{align*}
    [2n]\exp\frac{1}{2}\ln \frac{1}{1 - t^2}
     & = \sum_{i = 0}^{2n} \frac{1}{i!} \sum_{j_1 \dots j_n : \sum j_k = 2n} \frac{\prod \frac{j_k!}{\frac{j_k}{2}}}{\prod j_k!} \\
     & = \sum_{i = 0}^{2n} \frac{1}{i!} \sum_{j_1 \dots j_n : \sum j_k = 2n} \prod \frac{2}{j_k}                                 \\
    [2n]\exp \frac{1}{2} \ln \frac{1 + t}{1 - t}
     & = \sum_{i = 0}^{2n} \frac{1}{i!} \sum_{j_1 \dots j_n : \sum j_k = 2n} \frac{\prod 2(j_k - 1)!}{\prod j_k!}                \\
     & = \sum_{i = 0}^{2n} \frac{1}{i!} \sum_{j_1 \dots j_n : \sum j_k = 2n} \frac{\prod 2}{\prod j_k}                           \\
\end{align*}

\section{}
``Произведение с коробочкой'': Обозначим \(C = A^{\square} \times B\), как множество упорядоченных пар объектов из \(A\) и \(B\) со всеми возможными нумерациями, где атом с номером \(1\) принадлежит первому элементу пары. Выведите формулу для \(c_n\).

\[c_n = \sum_{k = 0}^n \binom{n - 1}{k - 1} a_k b_{n - k}\]

Т.к. только последние \(n - 1\) меток надо распределить, при этом \(k - 1\) первому.

\section{}
Докажите, что если \(C = A^{\square} \times B\), то \(C'(z) = A'(z) \cdot B(z)\).

\begin{align*}
    c_n      & = \sum_{k = 0}^n \binom{n - 1}{k - 1} a_k b_{n - k}          \\
             & = \frac{1}{n} \sum_{k = 0}^n \binom{n}{k} (ka_k) b_{n - k}   \\
             & = \frac{1}{n} \sum_{k = 0}^n \binom{n}{k} a_{k+1}' b_{n - k} \\
    nc_n     & = \sum_{k = 0}^n \binom{n}{k} a_{k+1}' b_{n - k}             \\
    c_{n+1}' & = \sum_{k = 0}^n \binom{n}{k} a_{k+1}' b_{n - k}             \\
\end{align*}

\section{}
Комбинаторный объект ``двоичная куча''. Рассмотрим помеченные двоичные деревья, где каждая вершина имеет двух детей, левого и правого (любое из этих поддеревьев может быть пустым), а также число в родителе вершины меньше числа в самой вершине (так, вершина с номером 1 --- всегда корень). Используя комбинаторную конструкцию ``произведение с коробочкой'', составьте и решите уравнение на экспоненциальную производящую функцию для двоичных куч.

\[T = 1 + \frac{1}{1 - t}^\square \times T \times T\]

\begin{itemize}
    \item 1: дерево может быть пустым.
    \item \(\frac{1}{1 - t}\) --- вершина.
    \item \(T \times T\) --- поддеревья
\end{itemize}

\begin{align*}
    T              & = 1 + \int T^2      \\
    T'             & = T^2               \\
    \frac{T'}{T^2} & = 1                 \\
    - \frac{1}{T}  & = T + C             \\
    - 1            & = T^2 + CT          \\
    T              & = \frac{1}{(1 - t)}
\end{align*}

\section{}
Обозначим за \(G(t)\) экспоненциальную производящую функцию всех помеченных графов. Чему равно \(g_n\)? Выразите производящую функцию связных помеченных графов, используя \(G(t)\).

\[g_n = 2^{\binom{n}{2}}\]

\begin{align*}
    G         & = Set(C \setminus \{\varepsilon\}) \\
    G         & = e^{C - 1}                        \\
    \ln G     & = C - 1                            \\
    1 + \ln G & = C                                \\
\end{align*}

\end{document}