\input{preamble.sty}

\begin{document}

\setcounter{section}{111}

\section{}
Возрастающе-убывающей перестановкой называется перестановка, которая поочередно возрастает и убывает: $x_1 < x_2 > x_3 < x_4 \ldots$. Обозначим количество возрастающе-убывающих перестановок размера $n$ как $a_n$. Докажите, что экспоненциальной производящей функцией для последовательности $a_n$ является $(1+\sin t)/\cos t$.

\subsection{Простое решение}

Рассмотрим перестановку длины \(n + 1\). Для неё выполняется такая рекурсия:
\[2 A_{n+1} = \sum_{k = 0}^n A_k A_{n - k}\]
и в случае \(n = 0\) база \(1\), в случае \(n = 1\) база тоже \(1\).

Тогда \(2 A' = A^2 + 1\). Решаем дифур, получаем искомое.

\subsection{\(\pi \leftrightarrow T\)}

Будем сопоставлять перестановкам деревья: \(x_1 \dots x_n \leftrightarrow\) корень \(\min_i x_i\), левое поддерево \(x_1 \dots x_{\arg\min_i x_i - 1}\) и правое поддерево \(x_1 \dots x_{\arg\min_i x_i - 1}\). Такие деревья возрастающие, т.е. \(a \to b \Rightarrow a < b\)

\subsection{\(n \equiv 1 \mod 2\)}

Несложно заметить, что у деревьев, соответствующим таким перестановкам, нет бамбуковых поддеревьев кроме \(T_1\) и \(T_2\) \textit{(и тогда последняя вершина --- лист)}.

\begin{align}
    A  & = t + t^{\square} \cdot (A \cdot A)    \\
    A' & = 1 + (t^{\square} \cdot (A \cdot A))' \\
    A' & = 1 + A^2                              \\
    A  & = \tg t                                \\
\end{align}

\subsection{\(n \equiv 1 \mod 2\)}

Аналогичными суждениями:

\begin{align}
    B  & = 1 + (t^\square \cdot A \cdot B) \\
    B' & = A \cdot B                       \\
    B' & = \tg t \cdot B                   \\
    B = \frac{1}{\cos t}
\end{align}

\section{}
Производящая функция Ньютона. Для последовательности $g_0, g_1, \ldots, g_n, \ldots$ производящая функция Ньютона определена как $\dot G(z) = \sum_n g_n{z \choose n}$. Пусть выполнено равенство: $\dot H(z) = \dot F(z) \cdot \dot G(z)$. Как связаны последовательности $f_i$, $g_i$ и $h_i$?

% \subsection{Дифференциальное дискретное исчисление}

Пусть \(\Delta f(x) = f(x + 1) - f(x)\), \(\Delta^n\) есть \(n\) композиций \(\Delta\). В терминах операторов \(\Delta^n = (S - 1)^n\), где \(S\) --- оператор сдвига \(f(x) \mapsto f(x + 1)\), а также действует обратное соотношение \(S^n = (\Delta + 1)^n\).

Заметим, что для ПФ Ньютона \(g_n = \Delta^n \dot{G}(0)\) по ряду, аналогичному ряду тейлора:
\[\dot{G}(a + x) = \sum_n \frac{\Delta^n \dot{G}(a)}{n!} x^n \]

\[\Delta^n \dot{F}(z) \cdot \dot{G}(z) = \sum_k \binom{n}{k} (\Delta^k S^{n - k} \dot{F}(z))(\Delta^{n - k} \dot{G}(z))\]
Это аналогично обобщенной формуле \(n\)-той производной двух функций, доказывается так же по индукции.
\[\Delta^n \dot{F}(z) \cdot \dot{G}(z) = \sum_k \binom{n}{k} (\Delta^k (\Delta + 1)^{n - k} \dot{F}(z))(\Delta^{n - k} \dot{G}(z))\]
\[\Delta^n \dot{F}(z) \cdot \dot{G}(z) = \sum_k \binom{n}{k} (\Delta^k \sum_j \binom{n - k}{j} \Delta^j \dot{F}(z))(\Delta^{n - k} \dot{G}(z))\]
\[h_n = \sum_k \binom{n}{k} \sum_j \binom{n - k}{j} f_{j + k} g_{n - k}\]

\section{}
Найдите ЭПФ для чисел Эйлера I рода

% Число Эйлера I рода \(A_{n, k}\) есть число перестановок \(n\) элементов с \(k\) подъемами. Классическим аргументом получаем:
% \[A_n(t) = \sum_{k = 0}^{n - 1}\]

\section{}
Найдите ЭПФ для чисел Эйлера II рода

Утром отпарсить!

\section{}
При решении задач этой серии можно при выражении использовать $\zeta(s)$. Обозначим как $\sigma_k(n)$ сумму по всем $d|n$ значений $d^k$. Найдите ПФД для $\sigma_1(n)$

\[\zeta(s) \cdot \zeta(s - 1)\]

\section{}
Найдите ПФД для $\sigma_k(n)$.

\[\zeta(s) \cdot \zeta(s - k)\]

\section{}
Найдите ПФД для последовательности $a_n = \sqrt{n}$.

\[\gamma(s - \frac{1}{2}) = \sum \frac{1}{n^{s - \frac{1}{2}}} = \sum \frac{\sqrt{n}}{n^s}\]

\section{}
Найдите ПФД для последовательности $a_n$, где $a_n = 1$ если $n$ квадрат целого числа, $a_n = 0$ иначе.

\section{}
Найдите ПФД для последовательности $a_n$, где $a_n = 1$ если $n$ свободно от квадратов, $a_n = 0$ иначе.

\[\frac{\zeta(s)}{\zeta(2s)} = \prod \frac{1 - p^{ - 2s}}{1 - p^{ - s}} = \prod 1 + p^{ - s} = \sum \frac{q(n)}{n^s}  \]
Т.к. \(q(n) = |\mu(n)|\)

\section{}
Зная ПФД для последовательности $a_n$, найдите ПФД для последовательности $a_n \cdot \ln n$.

\section{}
Докажите, что если $f(n)$  - мультипликативная функция, то $g(n) = \sum\limits_{d | n} f(d)$ тоже мультипликативна.

Если \(a, b\) не взаимно простые, то случай не содержателен. Пусть \(a, b\) взаимно простые.

\begin{align*}
    g(ab) & = \sum_{d | ab} f(d)                          \\
          & = \sum_{d_1 | a} \sum_{d_2 | b} f(d_1 d_2)    \\
          & = \sum_{d_1 | a} \sum_{d_2 | b} f(d_1) f(d_2) \\
          & = g(a)g(b) \\
\end{align*}

\section{}
Докажите, что свертка Дирихле двух мультипликативных функций мультипликативна.

\section{}
Докажите, что обратная по Дирихле функция к мультипликативной функции мультипликативна.

\section{}
Используя ПФД, докажите, что $\sum\limits_{d | n}\varphi(d) = n$

\section{}
Используя ПФД, докажите, что $\sum\limits_{d | n}\sigma_1(d)\varphi(n/d) = n \sigma_0(n)$.

\section{}
Назовем функцию полностью мультипликативной, если $f(ab) = f(a)f(b)$ для любых $a$ и $b$. Какие значения $f(n)$ достаточно задать, чтобы определить $f$ на всех положительных натуральных числах?

\section{}
Найдите ПФД для функции $\lambda(n) = (-1)^k$, где $k$ - количество простых делителей $n$ (с учетом кратности). Чему равна $\sum\limits_{d | n} \lambda(d)$?

\section{}
Рассмотрим строки из 0 и 1. Скажем, что строка $s$ периодичная, если ее можно представить как $k$ копий одной строки $p$: $s = p^k$. Выведите формулу для количества апериодичных строк для произвольного $n$. Указание: используйте формулу обращения Мебиуса.

\section{}
Найдите ПФД для последовательности $a_n = $ количество упорядоченных разбиений числа $n$ на (не обязательно простые) $k$ множителей, множитель 1 разрешен.

\[\zeta^k(s)\]
Это очевидно из продления определения умножения ПФД на \(k\) аргументов. Дописать!

\section{}
Найдите ПФД для последовательности $a_n = $ количество упорядоченных разбиений числа $n$ на $\ge 0$ (не обязательно простых) множителей, множитель 1 запрещен.

\begin{align*}
    \sum_n \frac{f(n)}{n^s} & = \sum_n \frac{\sum_k f_k(n)}{n^s}          \\
                            & = 1 + \sum_k (\zeta(s) - 1)^k               \\
                            & = 1 + \frac{\zeta(s) - 1}{1 - \zeta(s) + 1} \\
                            & = \frac{\zeta_1}{2 - \zeta(s)}              \\
\end{align*}

\section{}
Найдите ПФД для последовательности $a_n = 2^{\omega(n)}$, где $\omega(n)$ - количество различных простых делителей $n$.

\[\frac{\zeta^2(s)}{\zeta(2s)}\]

Возьмём результат задачи 120 и домножим на \(\zeta\). Таким образом, мы будем считать все безквадратные числа, являющиеся делителями \(n\). Эквивалентность этого и \(2^\omega\) --- известный факт в теории чисел, доказывается по индукции.
\end{document}
