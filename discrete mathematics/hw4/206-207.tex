\input{preamble.sty}

\setcounter{section}{205}

\begin{document}
\section{}
Пусть машине Тьюринга разрешено производить запись в каждую ячейку ленты только два раза: если значение в этой ячейке менялось уже дважды, запрещается записывать туда другой символ. Докажите, что такая модификация не меняет вычислительной мощности машины Тьюринга.

Будем симулировать один шаг обычной машины Тьюринга. Копируем все содержимое \(T\) ленты на нетронутую часть оной и заполняем пробелами то, что уже скопировали.
При этом при копировании мы будем изменять содержимое ленты по функции перехода автомата, получив \(T'\); также будем поддерживать в ячейке, лежащей после рабочей ленты, номер ячейки \(n\), на которой на данном шаге находилась бы настоящая машина Тьюринга.

\[\begin{tikzcd}
        {\dots T n \dots \dots} \\
        {.\, .\. \dots T' n' \dots}
        \arrow[from=1-1, to=2-1]
    \end{tikzcd}\]

\section{}
Пусть машине Тьюринга разрешено производить запись в каждую ячейку ленты только один раз: если значение в этой ячейке уже менялось, запрещается записывать туда другой символ. Докажите, что такая модификация не меняет вычислительной мощности машины Тьюринга.

Будем действовать аналогично предыдущему номеру. Теперь мы не можем занулять использованные ячейки, вместо этого у нас для каждой ячейки будет написан флаг в соседней с ней ячейке --- принадлежит ли ячейка рассматриваемому \(T\) или эта ячейка уже была использована на предыдущих итерациях \textit{(или не использована вообще)}.


\end{document}
