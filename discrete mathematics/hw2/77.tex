\documentclass[12pt, a4paper]{article}

%<*preamble>
% Math symbols
\usepackage{amsmath, amsthm, amsfonts, amssymb}
\usepackage{accents}
\usepackage{esvect}
\usepackage{mathrsfs}
\usepackage{mathtools}
\mathtoolsset{showonlyrefs}
\usepackage{cmll}
\usepackage{stmaryrd}
\usepackage{physics}
\usepackage[normalem]{ulem}
\usepackage{ebproof}
\usepackage{extarrows}

% Page layout
\usepackage{geometry, a4wide, parskip, fancyhdr}

% Font, encoding, russian support
\usepackage[russian]{babel}
\usepackage[sb]{libertine}
\usepackage{xltxtra}

% Listings
\usepackage{listings}
\lstset{basicstyle=\ttfamily,breaklines=true}
\setmonofont[Scale=MatchLowercase]{JetBrains Mono}

% Miscellaneous
\usepackage{array}
\usepackage{booktabs}\renewcommand{\arraystretch}{1.2}
\usepackage{calc}
\usepackage{caption}
\usepackage{subcaption}
\captionsetup{justification=centering,margin=2cm}
\usepackage{catchfilebetweentags}
\usepackage{enumitem}
\usepackage{etoolbox}
\usepackage{float}
\usepackage{lastpage}
\usepackage{minted}
\usepackage{svg}
\usepackage{wrapfig}
\usepackage{xcolor}
\usepackage[makeroom]{cancel}

\newcolumntype{L}{>{$}l<{$}}
    \newcolumntype{C}{>{$}c<{$}}
\newcolumntype{R}{>{$}r<{$}}

% Footnotes
\usepackage[hang]{footmisc}
\setlength{\footnotemargin}{2mm}
\makeatletter
\def\blfootnote{\gdef\@thefnmark{}\@footnotetext}
\makeatother

% References
\usepackage{hyperref}
\hypersetup{
    colorlinks,
    linkcolor={blue!80!black},
    citecolor={blue!80!black},
    urlcolor={blue!80!black},
}

% tikz
\usepackage{tikz}
\usepackage{tikz-cd}
\usetikzlibrary{arrows.meta}
\usetikzlibrary{decorations.pathmorphing}
\usetikzlibrary{calc}
\usetikzlibrary{patterns}
\usepackage{pgfplots}
\pgfplotsset{width=10cm,compat=1.9}
\newcommand\irregularcircle[2]{% radius, irregularity
    \pgfextra {\pgfmathsetmacro\len{(#1)+rand*(#2)}}
    +(0:\len pt)
    \foreach \a in {10,20,...,350}{
            \pgfextra {\pgfmathsetmacro\len{(#1)+rand*(#2)}}
            -- +(\a:\len pt)
        } -- cycle
}

\providetoggle{useproofs}
\settoggle{useproofs}{false}

\pagestyle{fancy}
\lhead{Лабораторная работа №2}
\lfoot{Михайлов Максим}
\rfoot{M3337}
\cfoot{}
\rhead{стр. \thepage\ из \pageref*{LastPage}}

\newcommand{\R}{\mathbb{R}}
\newcommand{\Q}{\mathbb{Q}}
\newcommand{\Z}{\mathbb{Z}}
\newcommand{\B}{\mathbb{B}}
\newcommand{\N}{\mathbb{N}}
\renewcommand{\Re}{\mathfrak{R}}
\renewcommand{\Im}{\mathfrak{I}}

\newcommand{\const}{\text{const}}
\newcommand{\cond}{\text{cond}}

\newcommand{\teormin}{\textcolor{red}{!}\ }

\DeclareMathOperator*{\xor}{\oplus}
\DeclareMathOperator*{\equ}{\sim}
\DeclareMathOperator{\sign}{\text{sign}}
\DeclareMathOperator{\Sym}{\text{Sym}}
\DeclareMathOperator{\Asym}{\text{Asym}}

\DeclarePairedDelimiter{\ceil}{\lceil}{\rceil}

% godel
\newbox\gnBoxA
\newdimen\gnCornerHgt
\setbox\gnBoxA=\hbox{$\ulcorner$}
\global\gnCornerHgt=\ht\gnBoxA
\newdimen\gnArgHgt
\def\godel #1{%
    \setbox\gnBoxA=\hbox{$#1$}%
    \gnArgHgt=\ht\gnBoxA%
    \ifnum     \gnArgHgt<\gnCornerHgt \gnArgHgt=0pt%
    \else \advance \gnArgHgt by -\gnCornerHgt%
    \fi \raise\gnArgHgt\hbox{$\ulcorner$} \box\gnBoxA %
    \raise\gnArgHgt\hbox{$\urcorner$}}

% \theoremstyle{plain}

\theoremstyle{definition}
\newtheorem{theorem}{Теорема}
\newtheorem*{definition}{Определение}
\newtheorem{axiom}{Аксиома}
\newtheorem*{axiom*}{Аксиома}
\newtheorem{lemma}{Лемма}
\newenvironment{solution}[1][Решение.]{\begin{proof}[#1]}{\end{proof}}

\theoremstyle{remark}
\newtheorem*{remark}{Примечание}
\newtheorem*{exercise}{Упражнение}
\newtheorem{corollary}{Следствие}[theorem]
\newtheorem*{statement}{Утверждение}
\newtheorem*{corollary*}{Следствие}
\newtheorem*{example}{Пример}
\newtheorem{observation}{Наблюдение}
\newtheorem*{prop}{Свойства}
\newtheorem*{obozn}{Обозначение}

% subtheorem
\makeatletter
\newenvironment{subtheorem}[1]{%
    \def\subtheoremcounter{#1}%
    \refstepcounter{#1}%
    \protected@edef\theparentnumber{\csname the#1\endcsname}%
    \setcounter{parentnumber}{\value{#1}}%
    \setcounter{#1}{0}%
    \expandafter\def\csname the#1\endcsname{\theparentnumber.\Alph{#1}}%
    \ignorespaces
}{%
    \setcounter{\subtheoremcounter}{\value{parentnumber}}%
    \ignorespacesafterend
}
\makeatother
\newcounter{parentnumber}

\newtheorem{manualtheoreminner}{Теорема}
\newenvironment{manualtheorem}[1]{%
    \renewcommand\themanualtheoreminner{#1}%
    \manualtheoreminner
}{\endmanualtheoreminner}

\newcommand{\dbltilde}[1]{\accentset{\approx}{#1}}
\newcommand{\intt}{\int\!}

% magical thing that fixes paragraphs
\makeatletter
\patchcmd{\CatchFBT@Fin@l}{\endlinechar\m@ne}{}
{}{\typeout{Unsuccessful patch!}}
\makeatother

\newcommand{\get}[2]{
    \ExecuteMetaData[#1]{#2}
}

\newcommand{\getproof}[2]{
    \iftoggle{useproofs}{\ExecuteMetaData[#1]{#2proof}}{}
}

\newcommand{\getwithproof}[2]{
    \get{#1}{#2}
    \getproof{#1}{#2}
}

\newcommand{\import}[3]{
    \subsection{#1}
    \getwithproof{#2}{#3}
}

\newcommand{\given}[1]{
    Дано выше. (\ref{#1}, стр. \pageref{#1})
}

\renewcommand{\ker}{\text{Ker }}
\newcommand{\im}{\text{Im }}
\renewcommand{\grad}{\text{grad}}
\newcommand{\rg}{\text{rg}}
\newcommand{\defeq}{\stackrel{\text{def}}{=}}
\newcommand{\defeqfor}[1]{\stackrel{\text{def } #1}{=}}
\newcommand{\itemfix}{\leavevmode\makeatletter\makeatother}
\newcommand{\?}{\textcolor{red}{???}}
\renewcommand{\emptyset}{\varnothing}
\newcommand{\longarrow}[1]{\xRightarrow[#1]{\qquad}}
\DeclareMathOperator*{\esup}{\text{ess sup}}
\newcommand\smallO{
    \mathchoice
    {{\scriptstyle\mathcal{O}}}% \displaystyle
    {{\scriptstyle\mathcal{O}}}% \textstyle
    {{\scriptscriptstyle\mathcal{O}}}% \scriptstyle
    {\scalebox{.6}{$\scriptscriptstyle\mathcal{O}$}}%\scriptscriptstyle
}
\renewcommand{\div}{\text{div}\ }
\newcommand{\rot}{\text{rot}\ }
\newcommand{\cov}{\text{cov}}

\makeatletter
\newcommand{\oplabel}[1]{\refstepcounter{equation}(\theequation\ltx@label{#1})}
\makeatother

\newcommand{\symref}[2]{\stackrel{\oplabel{#1}}{#2}}
\newcommand{\symrefeq}[1]{\symref{#1}{=}}

% xrightrightarrows
\makeatletter
\newcommand*{\relrelbarsep}{.386ex}
\newcommand*{\relrelbar}{%
    \mathrel{%
        \mathpalette\@relrelbar\relrelbarsep
    }%
}
\newcommand*{\@relrelbar}[2]{%
    \raise#2\hbox to 0pt{$\m@th#1\relbar$\hss}%
    \lower#2\hbox{$\m@th#1\relbar$}%
}
\providecommand*{\rightrightarrowsfill@}{%
    \arrowfill@\relrelbar\relrelbar\rightrightarrows
}
\providecommand*{\leftleftarrowsfill@}{%
    \arrowfill@\leftleftarrows\relrelbar\relrelbar
}
\providecommand*{\xrightrightarrows}[2][]{%
    \ext@arrow 0359\rightrightarrowsfill@{#1}{#2}%
}
\providecommand*{\xleftleftarrows}[2][]{%
    \ext@arrow 3095\leftleftarrowsfill@{#1}{#2}%
}

\allowdisplaybreaks

\newcommand{\unfinished}{\textcolor{red}{Не дописано}}

% Reproducible pdf builds 
\special{pdf:trailerid [
<00112233445566778899aabbccddeeff>
<00112233445566778899aabbccddeeff>
]}
%</preamble>


\DeclareMathOperator*{\argmin}{arg\,min}

\begin{document}

\section*{Условие}

Рассмотрим случайное блуждание точки на прямой, пусть точка начинает в точке $0$ и каждую секунду переходит равновероятно на $1$ влево или вправо. Докажите, что математическое ожидание максимума координаты точки за $n$ шагов есть $O(\sqrt n)$.

\subsubsection*{Формализация условия}

$\xi_i:=\pm 1, \mathbb P(\xi_i=1)=\mathbb P(\xi_i=-1)=0.5$ --- один шаг \textit{(он происходит раз в секунду)}.

$X_n$ --- координата спустя $n$ шагов:
$$X_n:=\sum_{i=1}^n \xi_i$$
$M_n$ --- максимум координаты спустя $n$ шагов:
$$M_n:=\max_{i\in[1, n]} X_i$$
Доказать: $\mathbb E (M_n)=\mathcal O(\sqrt n)$

\section*{Решение}

\begin{statement}
    $\mathbb P(M_n \geq a) = \mathbb P(X_n \geq a) + \mathbb P(X_n > a)$
\end{statement}
\begin{proof}
    Заметим, что система событий $\{X_n \geq a, X_n < a\}$ --- полная. Тогда по формуле полной вероятности:
    $$\mathbb P(M_n \geq a) = \mathbb P(M_n \geq a | X_n \geq a) \mathbb P(X_n \geq a) + \mathbb P(M_n \geq a | X_n < a) \mathbb P(X_n < a)$$
    $$(X_n \geq a \Rightarrow M_n \geq a) \Rightarrow \mathbb P(M_n \geq a | X_n \geq a)=1$$
    $$\mathbb P(M_n \geq a | X_n < a) \mathbb P(X_n < a)\stackrel{def}=\frac{\mathbb P(M_n \geq a \cap X_n < a)}{\mathbb P(X_n < a)}\mathbb P(X_n < a)=\mathbb P(M_n \geq a \cap X_n < a)$$
    Итого:
    $$\mathbb P(M_n \geq a) = \mathbb P(X_n \geq a) + \mathbb P(M_n \geq a \cap X_n < a)$$
    % Рассмотрим $M_n \geq a \cap X_n \leq a$. Заметим, что $M_n \geq a \Rightarrow \exists \tilde N\in[1, n] : X_N=a$, т.к. для достижения максимума $>a$ необходимо достичь $a$.

    Обозначим за $N$ первую секунду, такую что $X_N=a$:
    $$N:=\argmin_{i\in[1,\infty)}(x_i=a)$$
    $N$ существует с вероятностью $1$.

    Построим по последовательности шагов $\{\xi_i\}_{i=1}^n$ последовательность $\{\tilde \xi_i\}_{i=1}^n$, такую что:
    $$\tilde \xi_i=\begin{cases}
        \xi_i ,& i\leq N \\
        -\xi_i ,& i>N
    \end{cases}$$
    По построению $\{\tilde \xi_i\}$ совпадает с $\{\xi_i\}$ с начала и до точки $N$, после которой она зеркально отражена относительно оси $x$.
    
    Сопоставим $\{\tilde \xi_i\}$ последовательность префиксных сумм $\{\tilde X_i\}$ \textit{(аналогично $\{X_i\}$)}. $\{\tilde X_i\}$ совпадает с $\{X_i\}$ с начала и до точки $N$, после которой она зеркально отражена относительно горизонтальной прямой $y=a$.

    \clearpage

    \begin{figure}
        \includegraphics[scale=0.5]{imgs/wienerxd.png}
        \centering
        \caption{Пример \(X\) и $\tilde X$: $a=50, N\approx 2800$}
    \end{figure}

    \begin{statement}
        $$M_n \geq a \cap X_n < a \Leftrightarrow \tilde X_n > a$$
    \end{statement}
    \begin{proof}
        \begin{enumerate}
            \item Докажем ``$\Rightarrow$''
    
            $M_n\geq a \Rightarrow N \leq n$, иначе $a$ еще не было достигнуто.
    
            $X_n < a \Rightarrow N\not=n$, иначе противоречие с определением $N$ ($X_N=a$)
    
            Итого, $N < n \Rightarrow a-X_n=\tilde X_n - a$ \textit{(симметрия относительно $a$)}.
            $$X_n<a \Rightarrow a-X_n>0 \Rightarrow \tilde X_n - a > 0 \Rightarrow \tilde X_n > a$$

            \item Докажем ``$\Leftarrow$''
            
            $\tilde X_n > a \Rightarrow N < n$, т.к. $a$ было впервые достигнуто раньше.

            $N < n \Rightarrow M_n\geq a$ по тому же самому утверждению.

            Аналогично пункту c ``$\Rightarrow$'' доказывается $X_n < a$.
        \end{enumerate}
    \end{proof}

    Т.к. $M_n \geq a \cap X_n < a \Leftrightarrow \tilde X_n > a$, $\mathbb P(M_n \geq a \cap X_n < a)=\mathbb P(\tilde X_n > a)$.

    $\mathbb P(\tilde X_n > a)=\mathbb P(X_n > a)$, т.к. $\xi_i$ равновероятно распределены.

    Итого, $\mathbb P(M_n \geq a) = \mathbb P(X_n \geq a) + \mathbb P(M_n \geq a \cap X_n < a)=\mathbb P(X_n \geq a) + \mathbb P(X_n > a)$
\end{proof}

$$\mathbb E(M_n)=\sum_{i=1}^n P(M_n \geq i)=\sum_{i=1}^n \mathbb P(X_n \geq i) + \mathbb P(X_n > i)=\sum_{i=1}^n 2\mathbb P(X_n > i) + \mathbb P(X_n = i)$$

Заметим, что
$$\mathbb P(|X_n| > i)=\mathbb P(X_n > i)+\mathbb P(-X_n > i)$$
По симметрии блуждания точки:
$$\mathbb P(X_n > i)=\mathbb P(-X_n > i) \Rightarrow$$
$$\Rightarrow \mathbb P(|X_n| > i)=2\mathbb P(X_n > i)$$
Аналогичное утверждение верно для равенства:
$$\mathbb P(|X_n| = i)=2\mathbb P(X_n = i)$$
Подставим в $\mathbb E(M_n)$:
$$\mathbb E(M_n)=\sum_{i=1}^n \mathbb P(|X_n| > i) + \mathbb P(X_n = i)=\mathbb E(|X_n|)+\sum_{i=1}^n 0.5\mathbb P(|X_n| = i)\leq$$
$$\leq \mathbb E(|X_n|) + 0.5\sum_{i=1}^n i\mathbb P(|X_n| = i)=1.5\mathbb E(|X_n|)=\mathcal O(\sqrt n)$$
Последний переход доказан в предыдущем задании.

\end{document}