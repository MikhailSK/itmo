\input{preamble.sty}

\DeclareMathOperator*{\argmin}{arg\,min}

\begin{document}

\section*{Условие}

Рассмотрим случайное блуждание точки на прямой, пусть точка начинает в точке $0$ и каждую секунду переходит равновероятно на $1$ влево или вправо. Докажите, что математическое ожидание максимума координаты точки за $n$ шагов есть $O(\sqrt n)$.

\subsubsection*{Формализация условия}

$\xi_i:=\pm 1, \mathbb P(\xi_i=1)=\mathbb P(\xi_i=-1)=0.5$ --- один шаг \textit{(он происходит раз в секунду)}.

$X_n$ --- координата спустя $n$ шагов:
$$X_n:=\sum_{i=1}^n \xi_i$$
$M_n$ --- максимум координаты спустя $n$ шагов:
$$M_n:=\max_{i\in[1, n]} X_i$$
Доказать: $\mathbb E (M_n)=\mathcal O(\sqrt n)$

\section*{Решение}

\begin{statement}
    $\mathbb P(M_n \geq a) = \mathbb P(X_n \geq a) + \mathbb P(X_n > a)$
\end{statement}
\begin{proof}
    Заметим, что система событий $\{X_n \geq a, X_n < a\}$ --- полная. Тогда по формуле полной вероятности:
    $$\mathbb P(M_n \geq a) = \mathbb P(M_n \geq a | X_n \geq a) \mathbb P(X_n \geq a) + \mathbb P(M_n \geq a | X_n < a) \mathbb P(X_n < a)$$
    $$(X_n \geq a \Rightarrow M_n \geq a) \Rightarrow \mathbb P(M_n \geq a | X_n \geq a)=1$$
    $$\mathbb P(M_n \geq a | X_n < a) \mathbb P(X_n < a)\stackrel{def}=\frac{\mathbb P(M_n \geq a \cap X_n < a)}{\mathbb P(X_n < a)}\mathbb P(X_n < a)=\mathbb P(M_n \geq a \cap X_n < a)$$
    Итого:
    $$\mathbb P(M_n \geq a) = \mathbb P(X_n \geq a) + \mathbb P(M_n \geq a \cap X_n < a)$$
    % Рассмотрим $M_n \geq a \cap X_n \leq a$. Заметим, что $M_n \geq a \Rightarrow \exists \tilde N\in[1, n] : X_N=a$, т.к. для достижения максимума $>a$ необходимо достичь $a$.

    Обозначим за $N$ первую секунду, такую что $X_N=a$:
    $$N:=\argmin_{i\in[1,\infty)}(x_i=a)$$
    $N$ существует с вероятностью $1$.

    Построим по последовательности шагов $\{\xi_i\}_{i=1}^n$ последовательность $\{\tilde \xi_i\}_{i=1}^n$, такую что:
    $$\tilde \xi_i=\begin{cases}
        \xi_i ,& i\leq N \\
        -\xi_i ,& i>N
    \end{cases}$$
    По построению $\{\tilde \xi_i\}$ совпадает с $\{\xi_i\}$ с начала и до точки $N$, после которой она зеркально отражена относительно оси $x$.
    
    Сопоставим $\{\tilde \xi_i\}$ последовательность префиксных сумм $\{\tilde X_i\}$ \textit{(аналогично $\{X_i\}$)}. $\{\tilde X_i\}$ совпадает с $\{X_i\}$ с начала и до точки $N$, после которой она зеркально отражена относительно горизонтальной прямой $y=a$.

    \clearpage

    \begin{figure}
        \includegraphics[scale=0.5]{imgs/wienerxd.png}
        \centering
        \caption{Пример \(X\) и $\tilde X$: $a=50, N\approx 2800$}
    \end{figure}

    \begin{statement}
        $$M_n \geq a \cap X_n < a \Leftrightarrow \tilde X_n > a$$
    \end{statement}
    \begin{proof}
        \begin{enumerate}
            \item Докажем ``$\Rightarrow$''
    
            $M_n\geq a \Rightarrow N \leq n$, иначе $a$ еще не было достигнуто.
    
            $X_n < a \Rightarrow N\not=n$, иначе противоречие с определением $N$ ($X_N=a$)
    
            Итого, $N < n \Rightarrow a-X_n=\tilde X_n - a$ \textit{(симметрия относительно $a$)}.
            $$X_n<a \Rightarrow a-X_n>0 \Rightarrow \tilde X_n - a > 0 \Rightarrow \tilde X_n > a$$

            \item Докажем ``$\Leftarrow$''
            
            $\tilde X_n > a \Rightarrow N < n$, т.к. $a$ было впервые достигнуто раньше.

            $N < n \Rightarrow M_n\geq a$ по тому же самому утверждению.

            Аналогично пункту c ``$\Rightarrow$'' доказывается $X_n < a$.
        \end{enumerate}
    \end{proof}

    Т.к. $M_n \geq a \cap X_n < a \Leftrightarrow \tilde X_n > a$, $\mathbb P(M_n \geq a \cap X_n < a)=\mathbb P(\tilde X_n > a)$.

    $\mathbb P(\tilde X_n > a)=\mathbb P(X_n > a)$, т.к. $\xi_i$ равновероятно распределены.

    Итого, $\mathbb P(M_n \geq a) = \mathbb P(X_n \geq a) + \mathbb P(M_n \geq a \cap X_n < a)=\mathbb P(X_n \geq a) + \mathbb P(X_n > a)$
\end{proof}

$$\mathbb E(M_n)=\sum_{i=1}^n P(M_n \geq i)=\sum_{i=1}^n \mathbb P(X_n \geq i) + \mathbb P(X_n > i)=\sum_{i=1}^n 2\mathbb P(X_n > i) + \mathbb P(X_n = i)$$

Заметим, что
$$\mathbb P(|X_n| > i)=\mathbb P(X_n > i)+\mathbb P(-X_n > i)$$
По симметрии блуждания точки:
$$\mathbb P(X_n > i)=\mathbb P(-X_n > i) \Rightarrow$$
$$\Rightarrow \mathbb P(|X_n| > i)=2\mathbb P(X_n > i)$$
Аналогичное утверждение верно для равенства:
$$\mathbb P(|X_n| = i)=2\mathbb P(X_n = i)$$
Подставим в $\mathbb E(M_n)$:
$$\mathbb E(M_n)=\sum_{i=1}^n \mathbb P(|X_n| > i) + \mathbb P(X_n = i)=\mathbb E(|X_n|)+\sum_{i=1}^n 0.5\mathbb P(|X_n| = i)\leq$$
$$\leq \mathbb E(|X_n|) + 0.5\sum_{i=1}^n i\mathbb P(|X_n| = i)=1.5\mathbb E(|X_n|)=\mathcal O(\sqrt n)$$
Последний переход доказан в предыдущем задании.

\end{document}