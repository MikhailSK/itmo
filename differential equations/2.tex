\documentclass[12pt, a4paper]{article}

%<*preamble>
% Math symbols
\usepackage{amsmath, amsthm, amsfonts, amssymb}
\usepackage{accents}
\usepackage{esvect}
\usepackage{mathrsfs}
\usepackage{mathtools}
\mathtoolsset{showonlyrefs}
\usepackage{cmll}
\usepackage{stmaryrd}
\usepackage{physics}
\usepackage[normalem]{ulem}
\usepackage{ebproof}
\usepackage{extarrows}

% Page layout
\usepackage{geometry, a4wide, parskip, fancyhdr}

% Font, encoding, russian support
\usepackage[russian]{babel}
\usepackage[sb]{libertine}
\usepackage{xltxtra}

% Listings
\usepackage{listings}
\lstset{basicstyle=\ttfamily,breaklines=true}
\setmonofont[Scale=MatchLowercase]{JetBrains Mono}

% Miscellaneous
\usepackage{array}
\usepackage{booktabs}\renewcommand{\arraystretch}{1.2}
\usepackage{calc}
\usepackage{caption}
\usepackage{subcaption}
\captionsetup{justification=centering,margin=2cm}
\usepackage{catchfilebetweentags}
\usepackage{enumitem}
\usepackage{etoolbox}
\usepackage{float}
\usepackage{lastpage}
\usepackage{minted}
\usepackage{svg}
\usepackage{wrapfig}
\usepackage{xcolor}
\usepackage[makeroom]{cancel}

\newcolumntype{L}{>{$}l<{$}}
    \newcolumntype{C}{>{$}c<{$}}
\newcolumntype{R}{>{$}r<{$}}

% Footnotes
\usepackage[hang]{footmisc}
\setlength{\footnotemargin}{2mm}
\makeatletter
\def\blfootnote{\gdef\@thefnmark{}\@footnotetext}
\makeatother

% References
\usepackage{hyperref}
\hypersetup{
    colorlinks,
    linkcolor={blue!80!black},
    citecolor={blue!80!black},
    urlcolor={blue!80!black},
}

% tikz
\usepackage{tikz}
\usepackage{tikz-cd}
\usetikzlibrary{arrows.meta}
\usetikzlibrary{decorations.pathmorphing}
\usetikzlibrary{calc}
\usetikzlibrary{patterns}
\usepackage{pgfplots}
\pgfplotsset{width=10cm,compat=1.9}
\newcommand\irregularcircle[2]{% radius, irregularity
    \pgfextra {\pgfmathsetmacro\len{(#1)+rand*(#2)}}
    +(0:\len pt)
    \foreach \a in {10,20,...,350}{
            \pgfextra {\pgfmathsetmacro\len{(#1)+rand*(#2)}}
            -- +(\a:\len pt)
        } -- cycle
}

\providetoggle{useproofs}
\settoggle{useproofs}{false}

\pagestyle{fancy}
\lhead{Лабораторная работа №2}
\lfoot{Михайлов Максим}
\rfoot{M3337}
\cfoot{}
\rhead{стр. \thepage\ из \pageref*{LastPage}}

\newcommand{\R}{\mathbb{R}}
\newcommand{\Q}{\mathbb{Q}}
\newcommand{\Z}{\mathbb{Z}}
\newcommand{\B}{\mathbb{B}}
\newcommand{\N}{\mathbb{N}}
\renewcommand{\Re}{\mathfrak{R}}
\renewcommand{\Im}{\mathfrak{I}}

\newcommand{\const}{\text{const}}
\newcommand{\cond}{\text{cond}}

\newcommand{\teormin}{\textcolor{red}{!}\ }

\DeclareMathOperator*{\xor}{\oplus}
\DeclareMathOperator*{\equ}{\sim}
\DeclareMathOperator{\sign}{\text{sign}}
\DeclareMathOperator{\Sym}{\text{Sym}}
\DeclareMathOperator{\Asym}{\text{Asym}}

\DeclarePairedDelimiter{\ceil}{\lceil}{\rceil}

% godel
\newbox\gnBoxA
\newdimen\gnCornerHgt
\setbox\gnBoxA=\hbox{$\ulcorner$}
\global\gnCornerHgt=\ht\gnBoxA
\newdimen\gnArgHgt
\def\godel #1{%
    \setbox\gnBoxA=\hbox{$#1$}%
    \gnArgHgt=\ht\gnBoxA%
    \ifnum     \gnArgHgt<\gnCornerHgt \gnArgHgt=0pt%
    \else \advance \gnArgHgt by -\gnCornerHgt%
    \fi \raise\gnArgHgt\hbox{$\ulcorner$} \box\gnBoxA %
    \raise\gnArgHgt\hbox{$\urcorner$}}

% \theoremstyle{plain}

\theoremstyle{definition}
\newtheorem{theorem}{Теорема}
\newtheorem*{definition}{Определение}
\newtheorem{axiom}{Аксиома}
\newtheorem*{axiom*}{Аксиома}
\newtheorem{lemma}{Лемма}
\newenvironment{solution}[1][Решение.]{\begin{proof}[#1]}{\end{proof}}

\theoremstyle{remark}
\newtheorem*{remark}{Примечание}
\newtheorem*{exercise}{Упражнение}
\newtheorem{corollary}{Следствие}[theorem]
\newtheorem*{statement}{Утверждение}
\newtheorem*{corollary*}{Следствие}
\newtheorem*{example}{Пример}
\newtheorem{observation}{Наблюдение}
\newtheorem*{prop}{Свойства}
\newtheorem*{obozn}{Обозначение}

% subtheorem
\makeatletter
\newenvironment{subtheorem}[1]{%
    \def\subtheoremcounter{#1}%
    \refstepcounter{#1}%
    \protected@edef\theparentnumber{\csname the#1\endcsname}%
    \setcounter{parentnumber}{\value{#1}}%
    \setcounter{#1}{0}%
    \expandafter\def\csname the#1\endcsname{\theparentnumber.\Alph{#1}}%
    \ignorespaces
}{%
    \setcounter{\subtheoremcounter}{\value{parentnumber}}%
    \ignorespacesafterend
}
\makeatother
\newcounter{parentnumber}

\newtheorem{manualtheoreminner}{Теорема}
\newenvironment{manualtheorem}[1]{%
    \renewcommand\themanualtheoreminner{#1}%
    \manualtheoreminner
}{\endmanualtheoreminner}

\newcommand{\dbltilde}[1]{\accentset{\approx}{#1}}
\newcommand{\intt}{\int\!}

% magical thing that fixes paragraphs
\makeatletter
\patchcmd{\CatchFBT@Fin@l}{\endlinechar\m@ne}{}
{}{\typeout{Unsuccessful patch!}}
\makeatother

\newcommand{\get}[2]{
    \ExecuteMetaData[#1]{#2}
}

\newcommand{\getproof}[2]{
    \iftoggle{useproofs}{\ExecuteMetaData[#1]{#2proof}}{}
}

\newcommand{\getwithproof}[2]{
    \get{#1}{#2}
    \getproof{#1}{#2}
}

\newcommand{\import}[3]{
    \subsection{#1}
    \getwithproof{#2}{#3}
}

\newcommand{\given}[1]{
    Дано выше. (\ref{#1}, стр. \pageref{#1})
}

\renewcommand{\ker}{\text{Ker }}
\newcommand{\im}{\text{Im }}
\renewcommand{\grad}{\text{grad}}
\newcommand{\rg}{\text{rg}}
\newcommand{\defeq}{\stackrel{\text{def}}{=}}
\newcommand{\defeqfor}[1]{\stackrel{\text{def } #1}{=}}
\newcommand{\itemfix}{\leavevmode\makeatletter\makeatother}
\newcommand{\?}{\textcolor{red}{???}}
\renewcommand{\emptyset}{\varnothing}
\newcommand{\longarrow}[1]{\xRightarrow[#1]{\qquad}}
\DeclareMathOperator*{\esup}{\text{ess sup}}
\newcommand\smallO{
    \mathchoice
    {{\scriptstyle\mathcal{O}}}% \displaystyle
    {{\scriptstyle\mathcal{O}}}% \textstyle
    {{\scriptscriptstyle\mathcal{O}}}% \scriptstyle
    {\scalebox{.6}{$\scriptscriptstyle\mathcal{O}$}}%\scriptscriptstyle
}
\renewcommand{\div}{\text{div}\ }
\newcommand{\rot}{\text{rot}\ }
\newcommand{\cov}{\text{cov}}

\makeatletter
\newcommand{\oplabel}[1]{\refstepcounter{equation}(\theequation\ltx@label{#1})}
\makeatother

\newcommand{\symref}[2]{\stackrel{\oplabel{#1}}{#2}}
\newcommand{\symrefeq}[1]{\symref{#1}{=}}

% xrightrightarrows
\makeatletter
\newcommand*{\relrelbarsep}{.386ex}
\newcommand*{\relrelbar}{%
    \mathrel{%
        \mathpalette\@relrelbar\relrelbarsep
    }%
}
\newcommand*{\@relrelbar}[2]{%
    \raise#2\hbox to 0pt{$\m@th#1\relbar$\hss}%
    \lower#2\hbox{$\m@th#1\relbar$}%
}
\providecommand*{\rightrightarrowsfill@}{%
    \arrowfill@\relrelbar\relrelbar\rightrightarrows
}
\providecommand*{\leftleftarrowsfill@}{%
    \arrowfill@\leftleftarrows\relrelbar\relrelbar
}
\providecommand*{\xrightrightarrows}[2][]{%
    \ext@arrow 0359\rightrightarrowsfill@{#1}{#2}%
}
\providecommand*{\xleftleftarrows}[2][]{%
    \ext@arrow 3095\leftleftarrowsfill@{#1}{#2}%
}

\allowdisplaybreaks

\newcommand{\unfinished}{\textcolor{red}{Не дописано}}

% Reproducible pdf builds 
\special{pdf:trailerid [
<00112233445566778899aabbccddeeff>
<00112233445566778899aabbccddeeff>
]}
%</preamble>


\lhead{Домашнее задание №2}
\lfoot{Михайлов Максим}
\cfoot{}
\rfoot{M3237}

\begin{document}

\section{$xy'=y(\ln y - \ln x)$}
\begin{align*}
    xy'                                             & = y(\ln y - \ln x)                              \\
    xy'                                             & = y\left(\ln \left(\frac{y}{x}\right)\right)    \\
    t := \frac{y}{x}                                & \quad t' = \frac{y'x-y}{x^2} \quad y' = xt' + t \\
    x(xt'+t)                                        & = xt(\ln (t))                                   \\
    xt'+t                                           & = t\ln (t)                                      \\
    xt'                                             & = t(\ln (t) - 1)                        \tag{1} \\
    \frac{dt}{t(\ln (t) - 1)}                       & = \frac{dx}{x}                                  \\
    \int \frac{dt}{t(\ln (t) - 1)}                  & = \int \frac{dx}{x}                             \\
    \int \frac{dt}{t(\ln \left(\frac{t}{e}\right))} & = \ln x + C                                     \\
    \ln \left(\ln \left(\frac{t}{e}\right)\right)   & = \ln x + C                                     \\
    \ln \left(\frac{t}{e}\right)                    & = e^{\ln x}e^C                                  \\
    \ln \left(\frac{t}{e}\right)                    & = xe^C                                          \\
    \frac{t}{e}                                     & = e^{xe^C}                                      \\
    \frac{y}{x}                                     & = e^{xe^C+1}                                    \\
    y                                               & = xe^{xe^C+1}                                   \\
\end{align*}

(1): $\sphericalangle \ln t - 1 \equiv 0$. $y = ex$ --- подходит.

Ответ: $y = xe^{xe^C+1}$ или $y=ex$

\section{$2x+3y-5+(3x+2y-5)y'=0$}
\begin{align*}
    2x+3y-5+(3x+2y-5)y'     & = 0 \\
    (2x+3y-5)dx+(3x+2y-5)dy & = 0 \\
\end{align*}
$$\begin{cases}
        \cfrac{\partial (2x+3y-5)}{\partial y} = 3 \\
        \cfrac{\partial (3x+2y-5)}{\partial x} = 3 \\
    \end{cases} \Rightarrow \text{УПД}$$
$$\begin{cases}
        \cfrac{\partial u}{\partial x} = 2x+3y-5 \\
        \cfrac{\partial u}{\partial y} = 3x+2y-5 \\
    \end{cases}$$
\begin{align*}
    u & = \int (2x+3y-5)dx + C(y) \\
      & = -5x + x^2 + 3yx + C(y)
\end{align*}
\begin{align*}
    \frac{\partial u}{\partial y} & = 3x + C'(y)  \\
                                  & = 3x + 2y - 5
\end{align*}
\[C'(y) = 2y - 5 \Rightarrow C(y) = y^2 - 5y + C\]
\[x^2 + y^2 + 3xy - 5x - 5y + C = 0\]

\section{$y(1+\sqrt{x^2y^4+1})dx+2xdy=0$}

$y\equiv 0$ --- подходит.

\[y(1+\sqrt{x^2y^4+1})dx+2xdy=0\]
\[t := (xy^2)^2 \quad dt = 2xy^4dx + 4y^3x^2dy\]
\[\frac{dy}{dx} = \frac{\frac{dt}{dx}-2xy^4}{4y^3x^2} = \frac{\frac{dt}{dx} - \frac{t}{x}}{4t^{3/4}\sqrt x} = \frac{dt}{4t^{3/4}\sqrt xdx} - \frac{t^{1/4}}{4\sqrt x^3}\]
\begin{align*}
    \frac{t^{1/4}}{\sqrt x}(1+\sqrt{t+1})+2x\left(\frac{dt}{4t^{3/4}\sqrt xdx} - \frac{t^{1/4}}{4\sqrt x^3}\right) & = 0               \\
    \frac{t^{1/4}}{\sqrt x}(1+\sqrt{t+1})+\frac{dt\sqrt x}{2t^{3/4}dx} - \frac{t^{1/4}}{22\sqrt x}                 & = 0               \\
    \frac{1}{\sqrt x}\left(t^{1/4}(1+\sqrt{t+1}) +\frac{xdt}{2t^{3/4}dx} - \frac{t^{1/4}}{2} \right)               & = 0               \\
    \frac{1}{\sqrt x t^{3/4}}\left(t(1+\sqrt{t+1}) +\frac{xdt}{2dx} - \frac{t}{2} \right)                          & = 0               \\
    t(1+\sqrt{t+1}) +\frac{xdt}{2dx} - \frac{t}{2}                                                                 & = 0               \\
    \frac{t}{2} - t(1+\sqrt{t+1})                                                                                  & = \frac{xdt}{2dx} \\
\end{align*}
\begin{align*}
    \frac{dt}{dx}                                                   & = \frac{2}{x}\left(\frac{t}{2} - t(1+\sqrt{t+1})\right)         \\
    \frac{dt}{\frac{t}{2} - t(1+\sqrt{t+1})}                        & = \frac{2dx}{x}                                                 \\
    \int \frac{-dt}{t(1+2t\sqrt{t+1})}                              & = \int \frac{2dx}{x}                                            \\
    \int \frac{-dt}{t(1+2t\sqrt{t+1})}                              & = 2\ln |x| + C                                                  \\
    a := 1+2t\sqrt{t+1}                                             & \quad da = \frac{dt}{\sqrt{t+1}}  \quad t = \frac{(a-1)^2}{4}-1 \\
    -\int \frac{\frac{a-1}{2}da}{\left(\frac{(a-1)^2}{4}-1\right)a} & = 2\ln |x| + C                                                  \\
    -2\int \frac{(a-1)da}{a(a+1)(a-3)}                              & = 2\ln |x| + C                                                  \\
    -\int \frac{(a-1)da}{a(a+1)(a-3)}                               & = \ln |x| + C                                                   \\
\end{align*}
\begin{align*}
    -\int \left(\frac{1}{3a}+\frac{1}{6(a-3)}-\frac{1}{2(a+1)}\right)da & = \ln |x| + C \\
    -\ln |3a| - \ln |6(a-3)| + \ln |2(a+1)|                             & = \ln |x| + C \\
    -\ln |1+2t\sqrt{t+1}| -                                                             \\
    \ln |2t\sqrt{t+1}-2|                                                                \\
    + \ln |2+2t\sqrt{t+1}|                                              & = \ln |x| + C \\
    -\ln |1+2(xy^2)^2\sqrt{(xy^2)^2+1}| -                                               \\
    \ln |2(xy^2)^2\sqrt{(xy^2)^2+1}-2| +                                                \\
    \ln |2+2(xy^2)^2\sqrt{(xy^2)^2+1}|                                  & = \ln |x| + C \\
\end{align*}

Ответ: $y\equiv0$ или $\ln |2+2(xy^2)^2\sqrt{(xy^2)^2+1}| = \ln |x| + C$

\section{$4y^6+x^3=6xy^5y'$}

\begin{align*}
    y' & = \frac{4y^6+x^3}{6xy^5}              \\
    y' & = \frac{2y}{3x} + \frac{x^2}{6}y^{-5} \\
\end{align*}

Это уравнение Бернулли.

\begin{align*}
    t := y^{6}       & \quad t' = 6y^5y'                                                             \\
    4t+x^3           & = xt'                                                                         \\
    \frac{4t}{x}+x^2 & = t'                                                                          \\
    t                & = \left(C + \int x^2 e^{-\int \frac{4}{x}dx}dx \right) e^{\int \frac{4}{x}dx} \\
    t                & = \left(C + \int x^2 e^{-\ln x-C_1}dx \right) e^{C_1 + \ln x}                 \\
    t                & = \left(C + \int x e^{-C_1}dx \right) xe^{C_1}                                \\
    t                & = \left(C + \frac{1}{2}x^2 e^{-C_1} \right) xe^{C_1}                          \\
    y^6              & = \left(C + \frac{1}{2}x^2 e^{-C_1} \right) xe^{C_1}                          \\
\end{align*}

$y = \sqrt{-x}$ подходит.

Ответ: $y=\sqrt{-x}$ или $y^6 = \left(C + \frac{1}{2}x^2 e^{-C_1} \right) xe^{C_1}$

\section{}

% Самолёт А летит со скоростью 200 км/ч в направлении, наклон которого к горизонтальной линии равен ¾. Самолёт В из точки 50 км севернее вылетает на перехват его со скоростью 300 км/ч. Нос самолёта В непрерывно направлен к А. Найти уравнение траектории полёта самолёта В, наблюдаемой с борта самолёта А. Задачу решить: в прямоугольной системе координат и в полярной.

\section{}

Естественный прирост населения большого города пропорционален количеству жителей и промежутку  времени. Кроме того, население города увеличивается благодаря миграции: скорость прироста населения этим путём пропорциональна времени, отсчитываемому от момента, когда население города равнялось $A_0$. Найти зависимость числа жителей от времени (считать процесс непрерывным).

\begin{align*}
    dP                       & = k_1 dt P + k_2 dt       \\
    \frac{dP}{k_1P+k_2}      & = dt                      \\
    \frac{1}{k}\ln(k_1P+k_2) & = t + C                   \\
    k_1 P + k_2              & = C_1e^{k_1t}             \\
    k_1 A_0 + k_2            & = C_1                     \\
    k_1P + k_2               & = (k_1 A_0 + k_2)e^{k_1t}
\end{align*}

% До начала миграции:
% \begin{align*}
%     \frac{dA}{dt} & = k_1A                      \\
%     \frac{dA}{A}  & = k_1dt                     \\
%     \ln A         & = k_1t + C_1                \\
%     \ln A_0       & = k_1t_0 + C_1              \\
%     t_0           & = \frac{\ln A_0 - C_1}{k_1}
% \end{align*}

% После начала миграции:
% \begin{align*}
%     \frac{dA}{dt} & = k_1A + k_2(t-t_0) \\
%     % A             & = \left( C + \int k_2(t-t_0) e^{-\int k_1 dt} dt \right) e^{\int k_1 dt}                         \\
%     %               & = \left(C + k_2\int (t-t_0)e^{-k_1t - C_1} dt \right) e^{k_1t + C_1}                             \\
%     %               & = \left(C + e^{-C_1} k_2  \frac{e^{-k_1x}(t_0k_1 - k_1x - 1)}{k_1^2} \right) e^{k_1t + C_1}      \\
%     %               & = \left(C + e^{-C_1} k_2  \frac{e^{-k_1x}(\ln A_0 - C - k_1x - 1)}{k_1^2} \right) e^{k_1t + C_1} \\
% \end{align*}

% Решим методом Лагранжа:
% \[A = C(t) e^{\int k_1 dt} = C(t) e^{tk_1}\]
% \[A' = C'(t) e^{tk_1} + C(t)e^{tk_1} k_1\]
% \[C'(t) e^{tk_1} + C(t)e^{tk_1} k_1 = k_1 C(t) e^{tk_1} + k_2(t-t_0)\]
% \[C'(t) e^{tk_1} = k_2(t-t_0)\]
% \[\frac{dC}{dt}  = k_2(t-t_0) e^{-tk_1}\]
% \[C(t) = k_2\int (t-t_0)e^{-tk_1}dt \]
% \[C(t) = \frac{k_2e^{-tk_1} (t_0k_1 - k_1t - 1)}{k_1^2} + C\]
% \[A = \frac{k_2e^{-tk_1} (\ln A_0 - k_1t - 1)}{k_1^2} + C\]

\section{}

Капля сферической формы с начальной массой $M $ г, свободно падая в воздухе, равномерно испаряется и теряет ежесекундно $m$ г. Сила сопротивления воздуха пропорциональна произведению скорости капли на площадь её поверхности. Плотность жидкости гамма. Найти зависимость скорости движения капли от времени, прошедшего с начала её падения, в начальный момент времени скорость равна нулю. Принять, что коэффициент пропорциональности равен $k$.

Масса в момент времени $t$: $M-mt$.

По условию пропорциональности:
\begin{equation}
    F = kvS = 2kr^2\pi v
\end{equation}
По определению плотности:
\begin{equation}
    (M-mt)\rho = V = \frac{4}{3}\pi r^3 \Rightarrow r = \sqrt[3]{\frac{3}{4\pi} (M-mt)\rho}
\end{equation}
Подставим $(1)$ в $(2)$:
$$F = 2k \sqrt[3]{\frac{3}{4\pi} (M-mt)\rho}^2 \pi v$$
$$(M-mt)g - F = (M-mt)a$$
$$(M-mt)g - 2k \sqrt[3]{\frac{3}{4\pi} (M-mt)\rho}^2 \pi v = (M-mt) v'$$
$$g - 2k \sqrt[3]{\frac{3}{4\pi} \rho}^2 (M-mt)^{-1/3} \pi v = v'$$
\begin{align*}
    v & = \left( C + \int g\exp\left(\int 2k \sqrt[3]{\frac{3}{4\pi} \rho}^2 (M-mt)^{-1/3} \pi dt\right) dt \right) \exp\left(-\int 2k \sqrt[3]{\frac{3}{4\pi} \rho}^2 (M-mt)^{-1/3} \pi dt\right)
\end{align*}
\begin{align*}
    \int 2k \sqrt[3]{\frac{3}{4\pi} \rho}^2 (M-mt)^{-1/3} \pi dt & = 2k \sqrt[3]{\frac{3}{4\pi} \rho}^2 \pi \int (M-mt)^{-1/3} dt \\
                                                                 & = \frac{-3k}{m} \sqrt[3]{\frac{3}{4\pi} (M-mt) \rho}^2 \pi + C
\end{align*}
\begin{align*}
    v & = \left( C + \int g\exp\left( \frac{-3k}{m} \sqrt[3]{\frac{3}{4\pi} (M-mt) \rho}^2 \pi + C \right) dt \right) \exp\left(\frac{3k}{m} \sqrt[3]{\frac{3}{4\pi} (M-mt) \rho}^2 \pi + C\right)             \\
      & = \left( C + g \exp\left( \frac{-3k}{m} \pi \sqrt[3]{\frac{3}{4\pi} \rho}^2 \int (M-mt)^{2/3} dt + C \right) \right) \exp\left(\frac{3k}{m} \sqrt[3]{\frac{3}{4\pi} (M-mt) \rho}^2 \pi + C\right)      \\
      & = \left( C + g \exp\left( \frac{-3k}{m} \pi \sqrt[3]{\frac{3}{4\pi} \rho}^2 \frac{-3(M-mt)^{5/3}}{5m} + C \right) \right) \exp\left(\frac{3k}{m} \sqrt[3]{\frac{3}{4\pi} (M-mt) \rho}^2 \pi + C\right) \\
\end{align*}

\end{document}