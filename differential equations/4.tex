\documentclass[12pt, a4paper]{article}

%<*preamble>
% Math symbols
\usepackage{amsmath, amsthm, amsfonts, amssymb}
\usepackage{accents}
\usepackage{esvect}
\usepackage{mathrsfs}
\usepackage{mathtools}
\mathtoolsset{showonlyrefs}
\usepackage{cmll}
\usepackage{stmaryrd}
\usepackage{physics}
\usepackage[normalem]{ulem}
\usepackage{ebproof}
\usepackage{extarrows}

% Page layout
\usepackage{geometry, a4wide, parskip, fancyhdr}

% Font, encoding, russian support
\usepackage[russian]{babel}
\usepackage[sb]{libertine}
\usepackage{xltxtra}

% Listings
\usepackage{listings}
\lstset{basicstyle=\ttfamily,breaklines=true}
\setmonofont[Scale=MatchLowercase]{JetBrains Mono}

% Miscellaneous
\usepackage{array}
\usepackage{booktabs}\renewcommand{\arraystretch}{1.2}
\usepackage{calc}
\usepackage{caption}
\usepackage{subcaption}
\captionsetup{justification=centering,margin=2cm}
\usepackage{catchfilebetweentags}
\usepackage{enumitem}
\usepackage{etoolbox}
\usepackage{float}
\usepackage{lastpage}
\usepackage{minted}
\usepackage{svg}
\usepackage{wrapfig}
\usepackage{xcolor}
\usepackage[makeroom]{cancel}

\newcolumntype{L}{>{$}l<{$}}
    \newcolumntype{C}{>{$}c<{$}}
\newcolumntype{R}{>{$}r<{$}}

% Footnotes
\usepackage[hang]{footmisc}
\setlength{\footnotemargin}{2mm}
\makeatletter
\def\blfootnote{\gdef\@thefnmark{}\@footnotetext}
\makeatother

% References
\usepackage{hyperref}
\hypersetup{
    colorlinks,
    linkcolor={blue!80!black},
    citecolor={blue!80!black},
    urlcolor={blue!80!black},
}

% tikz
\usepackage{tikz}
\usepackage{tikz-cd}
\usetikzlibrary{arrows.meta}
\usetikzlibrary{decorations.pathmorphing}
\usetikzlibrary{calc}
\usetikzlibrary{patterns}
\usepackage{pgfplots}
\pgfplotsset{width=10cm,compat=1.9}
\newcommand\irregularcircle[2]{% radius, irregularity
    \pgfextra {\pgfmathsetmacro\len{(#1)+rand*(#2)}}
    +(0:\len pt)
    \foreach \a in {10,20,...,350}{
            \pgfextra {\pgfmathsetmacro\len{(#1)+rand*(#2)}}
            -- +(\a:\len pt)
        } -- cycle
}

\providetoggle{useproofs}
\settoggle{useproofs}{false}

\pagestyle{fancy}
\lhead{Лабораторная работа №2}
\lfoot{Михайлов Максим}
\rfoot{M3337}
\cfoot{}
\rhead{стр. \thepage\ из \pageref*{LastPage}}

\newcommand{\R}{\mathbb{R}}
\newcommand{\Q}{\mathbb{Q}}
\newcommand{\Z}{\mathbb{Z}}
\newcommand{\B}{\mathbb{B}}
\newcommand{\N}{\mathbb{N}}
\renewcommand{\Re}{\mathfrak{R}}
\renewcommand{\Im}{\mathfrak{I}}

\newcommand{\const}{\text{const}}
\newcommand{\cond}{\text{cond}}

\newcommand{\teormin}{\textcolor{red}{!}\ }

\DeclareMathOperator*{\xor}{\oplus}
\DeclareMathOperator*{\equ}{\sim}
\DeclareMathOperator{\sign}{\text{sign}}
\DeclareMathOperator{\Sym}{\text{Sym}}
\DeclareMathOperator{\Asym}{\text{Asym}}

\DeclarePairedDelimiter{\ceil}{\lceil}{\rceil}

% godel
\newbox\gnBoxA
\newdimen\gnCornerHgt
\setbox\gnBoxA=\hbox{$\ulcorner$}
\global\gnCornerHgt=\ht\gnBoxA
\newdimen\gnArgHgt
\def\godel #1{%
    \setbox\gnBoxA=\hbox{$#1$}%
    \gnArgHgt=\ht\gnBoxA%
    \ifnum     \gnArgHgt<\gnCornerHgt \gnArgHgt=0pt%
    \else \advance \gnArgHgt by -\gnCornerHgt%
    \fi \raise\gnArgHgt\hbox{$\ulcorner$} \box\gnBoxA %
    \raise\gnArgHgt\hbox{$\urcorner$}}

% \theoremstyle{plain}

\theoremstyle{definition}
\newtheorem{theorem}{Теорема}
\newtheorem*{definition}{Определение}
\newtheorem{axiom}{Аксиома}
\newtheorem*{axiom*}{Аксиома}
\newtheorem{lemma}{Лемма}
\newenvironment{solution}[1][Решение.]{\begin{proof}[#1]}{\end{proof}}

\theoremstyle{remark}
\newtheorem*{remark}{Примечание}
\newtheorem*{exercise}{Упражнение}
\newtheorem{corollary}{Следствие}[theorem]
\newtheorem*{statement}{Утверждение}
\newtheorem*{corollary*}{Следствие}
\newtheorem*{example}{Пример}
\newtheorem{observation}{Наблюдение}
\newtheorem*{prop}{Свойства}
\newtheorem*{obozn}{Обозначение}

% subtheorem
\makeatletter
\newenvironment{subtheorem}[1]{%
    \def\subtheoremcounter{#1}%
    \refstepcounter{#1}%
    \protected@edef\theparentnumber{\csname the#1\endcsname}%
    \setcounter{parentnumber}{\value{#1}}%
    \setcounter{#1}{0}%
    \expandafter\def\csname the#1\endcsname{\theparentnumber.\Alph{#1}}%
    \ignorespaces
}{%
    \setcounter{\subtheoremcounter}{\value{parentnumber}}%
    \ignorespacesafterend
}
\makeatother
\newcounter{parentnumber}

\newtheorem{manualtheoreminner}{Теорема}
\newenvironment{manualtheorem}[1]{%
    \renewcommand\themanualtheoreminner{#1}%
    \manualtheoreminner
}{\endmanualtheoreminner}

\newcommand{\dbltilde}[1]{\accentset{\approx}{#1}}
\newcommand{\intt}{\int\!}

% magical thing that fixes paragraphs
\makeatletter
\patchcmd{\CatchFBT@Fin@l}{\endlinechar\m@ne}{}
{}{\typeout{Unsuccessful patch!}}
\makeatother

\newcommand{\get}[2]{
    \ExecuteMetaData[#1]{#2}
}

\newcommand{\getproof}[2]{
    \iftoggle{useproofs}{\ExecuteMetaData[#1]{#2proof}}{}
}

\newcommand{\getwithproof}[2]{
    \get{#1}{#2}
    \getproof{#1}{#2}
}

\newcommand{\import}[3]{
    \subsection{#1}
    \getwithproof{#2}{#3}
}

\newcommand{\given}[1]{
    Дано выше. (\ref{#1}, стр. \pageref{#1})
}

\renewcommand{\ker}{\text{Ker }}
\newcommand{\im}{\text{Im }}
\renewcommand{\grad}{\text{grad}}
\newcommand{\rg}{\text{rg}}
\newcommand{\defeq}{\stackrel{\text{def}}{=}}
\newcommand{\defeqfor}[1]{\stackrel{\text{def } #1}{=}}
\newcommand{\itemfix}{\leavevmode\makeatletter\makeatother}
\newcommand{\?}{\textcolor{red}{???}}
\renewcommand{\emptyset}{\varnothing}
\newcommand{\longarrow}[1]{\xRightarrow[#1]{\qquad}}
\DeclareMathOperator*{\esup}{\text{ess sup}}
\newcommand\smallO{
    \mathchoice
    {{\scriptstyle\mathcal{O}}}% \displaystyle
    {{\scriptstyle\mathcal{O}}}% \textstyle
    {{\scriptscriptstyle\mathcal{O}}}% \scriptstyle
    {\scalebox{.6}{$\scriptscriptstyle\mathcal{O}$}}%\scriptscriptstyle
}
\renewcommand{\div}{\text{div}\ }
\newcommand{\rot}{\text{rot}\ }
\newcommand{\cov}{\text{cov}}

\makeatletter
\newcommand{\oplabel}[1]{\refstepcounter{equation}(\theequation\ltx@label{#1})}
\makeatother

\newcommand{\symref}[2]{\stackrel{\oplabel{#1}}{#2}}
\newcommand{\symrefeq}[1]{\symref{#1}{=}}

% xrightrightarrows
\makeatletter
\newcommand*{\relrelbarsep}{.386ex}
\newcommand*{\relrelbar}{%
    \mathrel{%
        \mathpalette\@relrelbar\relrelbarsep
    }%
}
\newcommand*{\@relrelbar}[2]{%
    \raise#2\hbox to 0pt{$\m@th#1\relbar$\hss}%
    \lower#2\hbox{$\m@th#1\relbar$}%
}
\providecommand*{\rightrightarrowsfill@}{%
    \arrowfill@\relrelbar\relrelbar\rightrightarrows
}
\providecommand*{\leftleftarrowsfill@}{%
    \arrowfill@\leftleftarrows\relrelbar\relrelbar
}
\providecommand*{\xrightrightarrows}[2][]{%
    \ext@arrow 0359\rightrightarrowsfill@{#1}{#2}%
}
\providecommand*{\xleftleftarrows}[2][]{%
    \ext@arrow 3095\leftleftarrowsfill@{#1}{#2}%
}

\allowdisplaybreaks

\newcommand{\unfinished}{\textcolor{red}{Не дописано}}

% Reproducible pdf builds 
\special{pdf:trailerid [
<00112233445566778899aabbccddeeff>
<00112233445566778899aabbccddeeff>
]}
%</preamble>


\lhead{Домашнее задание №4}
\lfoot{Михайлов Максим}
\cfoot{}
\rfoot{M3237}

\begin{document}

\section{Найти кривые, у которых радиус кривизны обратно пропорционален косинусу угла между касательной и осью абсцисс.}

\begin{align*}
    R                                             & = \frac{(r^2+r'^2)^{3/2}}{|r^2 + 2r'^2 - rr''|} \\
    \tg \alpha                                    & = \frac{r}{r'}                                  \\
    \tg^2 \alpha                                  & = \frac{r^2}{r'^2}                              \\
    \frac{1 - \cos^2 \alpha}{\cos^2 \alpha}       & = \frac{r^2}{r'^2}                              \\
    \frac{1}{\cos^2 \alpha} - 1                   & = \frac{r^2}{r'^2}                              \\
    \frac{1}{\cos^2 \alpha}                       & = \frac{r^2+r'^2}{r'^2}                         \\
    \frac{1}{\cos \alpha}                         & = \frac{\sqrt{r^2+r'^2}}{r'}                    \\
    \frac{(r^2+r'^2)^{3/2}}{|r^2 + 2r'^2 - rr''|} & = \frac{\sqrt{r^2+r'^2}}{r'}                    \\
    \frac{r^2+r'^2}{|r^2 + 2r'^2 - rr''|}         & = \frac{1}{r'}                                  \\
    t := r'                                       & \quad r'' = t't                                 \\
    \frac{r^2+t^2}{|r^2 + 2t^2 - rt't|}           & = \frac{1}{t}                                   \\
    r^2t+t^3                                      & = |r^2 + 2t^2 - rt't|                           \\
    \pm r^2t\pm t^3                               & = r^2 + 2t^2 - rt't                             \\
\end{align*}

Не решается, попробуем в декартовых координатах.

\begin{align*}
    R                                         & = \frac{(1+y'^2)^{3/2}}{|y''|} \\
    \frac{1}{\cos^2 \alpha} - 1               & = y'^2                         \\
    \frac{1}{\cos \alpha}                     & = \sqrt{y'^2 + 1}              \\
    R                                         & = \frac{n}{\cos \alpha}        \\
    \frac{(1+y'^2)^{3/2}}{|y''|}              & = n\sqrt{y'^2 + 1}             \\
    1+y'^2                                    & = |y''|n                       \\
    \frac{1}{n}                               & = \frac{|y''|}{1+y'^2}         \\
    \pm\frac{1}{n}                            & = \frac{y''}{1+y'^2}           \\
    \pm\frac{1}{n}                            & = \arctg' y'                   \\
    \int \pm\frac{1}{n} dx                    & = \int \arctg' y' dx           \\
    \pm\frac{x}{n} + C                        & = \arctg y'                    \\
    \tg\left(\pm\frac{x}{n} + C\right)        & = y'                           \\
    \int \tg\left(\pm\frac{x}{n} + C\right)dx & = \int dy                      \\
    \int \tg\left(\pm\frac{x}{n} + C\right)dx & = y + C_1                      \\
\end{align*}

Найдём этот интеграл:

\begin{align*}
    t := \pm\frac{x}{n} + C                   & \quad t' = \pm\frac{1}{n} \\
    \int \tg\left(\pm\frac{x}{n} + C\right)dx & = \int \pm \tg (t) n dt   \\
                                              & = \mp n\ln|\cos x|        \\
\end{align*}

Ответ: $y = \mp n\ln|\cos x| - C_1$

% \begin{align*}
%     R                                                & = \frac{|y''|}{(1+y'^2)^{3/2}}    \\
%     \tg \alpha                                       & = y'                              \\
%     \frac{1}{\cos^2 \alpha} - 1                      & = y'^2                            \\
%     \frac{1}{\cos \alpha}                            & = \sqrt{y'^2 + 1}                 \\
%     \frac{|y''|}{(1+y'^2)^{3/2}}                     & = \sqrt{y'^2 + 1}                 \\
%     |y''|                                            & = (y'^2 + 1)^2                    \\
%     t := y'                                          & \quad y'' = t't                   \\
%     |t't|                                            & = (t^2 + 1)^2                     \\
%     t't                                              & = \pm (t^2 + 1)^2                 \\
%     \frac{t't}{\pm (t^2 + 1)^2}                      & = 1                               \\
%     \int \frac{tdt}{\pm (t^2 + 1)^2}                 & = \int 1 dy                       \\
%     \int \frac{d(t^2 + 1)}{\pm 2(t^2 + 1)^2}         & = y + C                           \\
%     \mp\frac{1}{2(t^2 + 1)}                          & = y + C                           \\
%     \mp\frac{1}{2(y'^2 + 1)}                         & = y + C                 \tag{1}   \\
%     \mp\frac{1}{2(y + C)} - 1                        & = y'^2                            \\
%     \sqrt{\mp\frac{1}{2(y + C)} - 1}                 & = y'                      \tag{2} \\
%     \int \frac{dy}{\sqrt{\mp\frac{1}{2(y + C)} - 1}} & = x                               \\
% \end{align*}

% Найдём этот интеграл:

% \begin{align*}
%     \int \frac{dy}{\sqrt{\mp\frac{1}{2(y + C)} - 1}}
%      & =  \sqrt{2} \int \frac{dy}{\sqrt{\mp\frac{1}{y + C} - 2}} \\
% \end{align*}
% \begin{align*}
%     a := \sqrt{\frac{\mp 1}{y+C} - 2} \quad a' = \frac{\pm 1}{2(y + C)^2\sqrt{\frac{\mp 1}{y+C} - 2}} = \frac{\pm 1}{2(y + C)^2a} \\
% \end{align*}
% \begin{align*}
%     \sqrt{2} \int \frac{dy}{\sqrt{\mp\frac{1}{y + C} - 2}}
%      & = \sqrt 2 \int \frac{\pm 2(y + C)^2a da}{a}                                                                                                                                                  \\
%      & = \sqrt 2 \int \pm 2(y + C)^2 da                                                                                                                                                             \\
%      & = 2\sqrt 2 \int \pm \frac{da}{(a^2+2)^2}                                                                                                                                                     \\
%      & = \pm \frac{\sqrt 2}{2} \left(\frac{a}{a^2+2} + \int \frac{da}{a^2+2}\right)                                                                                                                 \\
%      & = \pm \frac{\sqrt 2}{2} \left(\frac{a}{a^2+2} + \frac{\arctg\left(\frac{a}{\sqrt 2}\right)}{\sqrt 2} + C_1 \right)                                                                           \\
%      & = \pm \frac{\sqrt 2}{2} \left(\frac{\sqrt{\frac{\mp 1}{y+C} - 2}}{\frac{\mp 1}{y+C} - 2 + 2} + \frac{\arctg\left(\frac{\sqrt{\frac{\mp 1}{y+C} - 2}}{\sqrt 2}\right)}{\sqrt 2} + C_1 \right) \\
%      & = \pm \frac{\sqrt 2}{2} \left(\mp(y+C)\sqrt{\frac{\mp 1}{y+C} - 2} + \frac{\arctg\left(\frac{\sqrt{\frac{\mp 1}{y+C} - 2}}{\sqrt 2}\right)}{\sqrt 2} + C_1 \right)                           \\
% \end{align*}

% (1): для $y\equiv C$ радиус кривизны не определен (бесконечен), поэтому это не решение.

% (2): надо проверить, что под корнем $\ge 0$

\section{Определить форму равновесия нерастяжимой нити с закрепленными концами, на которую действует нагрузка так, что на каждую единицу длины нагрузка одинакова (цепи цепного моста). Весом самой нити пренебречь.}

\begin{tikzpicture}
    \draw[-{>[length=2mm,width=3mm]}] (-1, 0) -- (17, 0) node[anchor=south west] {$x$};
    \draw[-{>[length=2mm,width=3mm]}] (0, -5) -- (0, 1) node[anchor=south east] {$y$};
    \draw (0, 0) .. controls (7, -3) .. (14, 0);
    \draw[blue, very thick] (9, -1.95) -- (10, -1.65) node[midway, above] {$\Delta l$};
    \node (part) at (9.5, -1.8) {};

    \draw[-{>[length=2mm,width=3mm]}] ($ (part) $) -- ($ (part) - (0, 2) $) node[anchor=west] {$\vec N$};
    \draw[dotted] (10, -1.65) -- (10, 0) node[anchor=south] {$x+dl$};
    \draw[dotted] (9, -1.95) -- (9, 0) node[anchor=south] {$x$};
    \draw[-{>[length=2mm,width=3mm]}] ($ (part) $) -- ($ (part) + (3, 1) $) node[anchor=south] {$\vec F_{x+\Delta l}$};
    \draw[-{>[length=2mm,width=3mm]}] ($ (part) $) -- ($ (part) - (3, 0.7) $) node[anchor=east] {$\vec F_{x}$};

    \node (cross1) at ($ (part) + (7.5, 1.85)$) {};
    \draw[dashed] ($ (part) $) -- (cross1);
    \draw[] ($ (part) + (4, 1) $) arc (12:15:5cm);
    \node at ($ (part) + (4, 0.7) $) {\small{$\Delta\alpha$}};
    \draw[] ($ (part) - (0, 1) $) arc (-90:-14:2cm);
    \node at ($ (part) - (-1.5, 1) $) {\small{$\frac{\pi}{2} + \alpha$}};

    \node (cross2) at ($ (part) + (5.7, 1.85)$) {};
    \draw[dashed] ($ (part) $) -- (cross2);
    \draw[] ($ (cross1) - (1, 0) $) arc (180:195:1cm);
    \node at ($ (cross1) + (-1, 0.3) $) {\small{$\alpha$}};
\end{tikzpicture}

Обозначения: $\vec F$ --- сила натяжения нити, $\vec N$ --- нагрузка на отрезок, $\alpha$ --- угол, $\Delta l$ --- длина рассматриваемого отрезка, $x$ --- координата начала отрезка, $\rho$ --- удельная плотность нити.

Т.к. нить в состоянии равновесия, $\vec a = \vec 0$.

Т.к. нагрузка распределена равномерно, $|N| = \rho \Delta l$, где $\rho$ --- удельная нагрузка.

По второму закону Ньютона:

\begin{align*}
    \vec F_x + \vec F_{x+\Delta l} + \vec N = \vec 0
\end{align*}

Спроецируем перпендикулярно $\vec F_{x}$:

\begin{align*}
    |F_{x+\Delta l}| \sin (\Delta\alpha) & = |N| \cos\alpha                             \\
    |F_{x+\Delta l}|                     & = \frac{|N| \cos\alpha}{\sin (\Delta\alpha)} \\
\end{align*}

Спроецируем перпендикулярно $\vec F_{x+dl}$:

\begin{align*}
    |F_{x}| \sin (\Delta\alpha) & = |N| \cos(\alpha + \Delta\alpha)                             \\
    |F_{x}|                     & = \frac{|N| \cos(\alpha + \Delta\alpha)}{\sin (\Delta\alpha)} \\
\end{align*}

Спроецируем параллельно $\vec F_x$:

\begin{align*}
    |F_x| + |N|\cos\left(\frac{\pi}{2} + \alpha\right)                                                       & = |F_{x+\Delta l}|\cos(\Delta\alpha)                           \\
    \frac{|N| \cos(\alpha + \Delta\alpha)}{\sin (\Delta\alpha)} + |N|\cos\left(\frac{\pi}{2} + \alpha\right) & = \frac{|N| \cos\alpha}{\sin (\Delta\alpha)}\cos(\Delta\alpha) \\
    \frac{\cos(\alpha + \Delta\alpha)}{\sin (\Delta\alpha)} -\sin\alpha                                      & = \frac{\cos\alpha}{\sin (\Delta\alpha)}\cos(\Delta\alpha)     \\
    \cos(\alpha + \Delta\alpha) -\sin\alpha\sin (\Delta\alpha)                                               & = \cos\alpha\cos(\Delta\alpha)                                 \\
    \cos(\alpha + \Delta\alpha)                                                                              & = \cos(\alpha - \Delta\alpha)                                  \\
\end{align*}

???

Не работает, будем проецировать на оси:

На $x$:

\begin{align*}
    F_{x+\Delta l} \cos(\alpha + \Delta\alpha)                                    & = F_x \cos \alpha                  \\
    \frac{F_{x+\Delta l} \cos(\alpha + \Delta\alpha)}{\Delta l}                   & = \frac{F_x \cos \alpha}{\Delta l} \\
    \frac{F_{x+\Delta l} \cos(\alpha + \Delta\alpha) - F_x \cos \alpha}{\Delta l} & = 0                                \\
    (F_{x}\cos\alpha)'                                                            & = 0                                \\
    F_{x}\cos\alpha                                                               & = \const = C                       \\
\end{align*}

На $y$:

\begin{align*}
    F_{x+dl} \sin(\alpha + \Delta\alpha)                                    & = F_x \sin \alpha + N \\
    F_{x+dl} \sin(\alpha + \Delta\alpha) - F_x \sin \alpha                  & = \rho \Delta l       \\
    \frac{F_{x+dl} \sin(\alpha + \Delta\alpha) - F_x \sin \alpha}{\Delta l} & = \rho                \\
    (F_x \sin \alpha)'                                                      & = \rho                \\
    \int (F_x \sin \alpha)' dx                                              & = \int \rho dx        \\
    F_x \sin \alpha                                                         & = \rho x + C_1        \\
    \frac{C}{\cos \alpha} \sin \alpha                                       & = \rho x + C_1        \\
    C \tg \alpha                                                            & = \rho x + C_1        \\
\end{align*}


\section{$y'=e^{xy'/y}$}

\begin{align*}
    y'                 & = e^{xy'/y}                                                 \\
    t := \frac{xy'}{y} & \quad y' = \frac{yt}{x}                                     \\
    \frac{yt}{x}       & = e^t                                                       \\
    y                  & = \frac{xe^t}{t}                                            \\
    e^t = y'           & = \left(\frac{xe^t}{t}\right)'                              \\
                       & = \frac{e^t}{t} + x\left(\frac{e^t}{y}\right)'              \\
                       & = \frac{e^t}{t} + x\left(\frac{te^tt' - e^t t'}{t^2}\right) \\
                       & = \frac{e^t}{t} + x\frac{e^tt'}{t^2}\left(t-1\right)        \\
    e^t                & = \frac{e^t}{t} + x\frac{e^tt'}{t^2}\left(t-1\right)        \\
    1                  & = \frac{1}{t} + x\frac{t'}{t^2}\left(t-1\right)             \\
    t                  & = 1 + x\frac{t'}{t}\left(t-1\right)                         \\
    t - 1              & = x\frac{t'}{t}\left(t-1\right)                     \tag{1} \\
    1                  & = x\frac{t'}{t}                                             \\
    t                  & = xt'                                                       \\
    t                  & = Cx                                                        \\
    y                  & = \frac{xe^{Cx}}{Cx}                                        \\
    y                  & = \frac{e^{Cx}}{C}                                          \\
\end{align*}

В силу (1) надо рассмотреть случай $t=1$:
\begin{align*}
    y & = \frac{xe^t}{t} \\
    y & = xe             \\
\end{align*}

Ответ: $y = \cfrac{e^{Cx}}{C}$ или $y = xe$

% \begin{align*}
%     y'               & = \exp\left(\frac{xy'}{y}\right)                \\
%     t := \frac{y}{x} & \quad t' = \frac{y'x-y}{x^2} \quad y' = xt' + t \\
%     xt' + t          & = \exp\left(\frac{xt' + t}{t}\right)            \\
% \end{align*}

\section{$xy'-y=\ln y$}

\begin{align*}
    xy'-y                     & = \ln y       \\
    xy'                       & = y + \ln y   \\
    \frac{y'}{y + \ln y}      & = \frac{1}{x} \\
    \int \frac{dy}{y + \ln y} & = \ln |x| + C \\
\end{align*}

Этот интеграл не выражается в элементарных функциях.

% Покажем, что $\int \cfrac{dy}{y + \ln y}$ не вычислим в терминах элементарных функций:

% \begin{align*}
%     \int \frac{dy}{y + \ln y} & = \int \ln\exp\frac{1}{y+\ln y} dy \\
%                               & = \ln\int\exp\frac{1}{y+\ln y}dy
% \end{align*}

% \[t := \ln y, dt = \frac{dy}{y}\]

\section{$yy''=2xy'^2 \quad y(2)=2 \quad y'(2)=0.5$}

\begin{align*}
    yy''                       & = 2xy'^2                                            \\
    t := \frac{y'}{y} \quad t' & = \frac{yy'' - y'^2}{y^2} \quad yy'' = t'y^2 + y'^2 \\
    t'y^2 + y'^2               & = 2xy'^2                                   \tag{1}  \\
    t' + \frac{y'^2}{y^2}      & = 2x\frac{y'^2}{y^2}                       \tag{2}  \\
    t' + t^2                   & = 2xt^2                                             \\
    t'                         & = t^2(2x - 1)                                       \\
    \frac{t'}{t^2}             & = 2x - 1                                            \\
    \int \frac{dt}{t^2}        & = \int (2x - 1)dx                                   \\
    -\frac{1}{t}               & = x^2 - x + C                                       \\
    -\frac{y}{y'}              & = x^2 - x + C                                       \\
    -\frac{2}{0.5}             & = 4 - 2 + C                                \tag{3}  \\
    -4                         & = 4 - 2 + C                                         \\
    -6                         & = C                                                 \\
    -\frac{y}{y'}              & = x^2 - x - 6                                       \\
    -\frac{y'}{y}              & = \frac{1}{x^2 - x - 6}                             \\
    \int-\frac{dy}{y}          & = \int\frac{dx}{x^2 - x - 6}                        \\
    -\ln|y| + C                & = \int\frac{dx}{x^2 - x - 6}                        \\
\end{align*}

Разобьем $\cfrac{1}{x^2 - x - 6}$ на сумму дробей:

\begin{align*}
    \frac{1}{(x-3)(x+2)} & = \frac{A}{x-3} + \frac{B}{x+2} \\
    1                    & = A(x+2) + B(x-3)               \\
    A = \frac{1}{5}      & \quad B = -\frac{1}{5}
\end{align*}

\begin{align*}
    -\ln|y| + C & = \int \left(\frac{1}{5(x-3)} - \frac{1}{5(x+2)}\right)dx \\
    -\ln|y| + C & = \frac{1}{5}\left(\ln|x-3| - \ln|x+2|\right)             \\
    -\ln|2| + C & = \frac{1}{5}\left(\ln|2-3| - \ln|2+2|\right)  \tag{4}    \\
    -\ln2 + C   & = \frac{1}{5}\left(\ln1 - \ln4\right)                     \\
    -5\ln2 + 5C & = - 2\ln2                                                 \\
    5C          & = 3\ln2                                                   \\
    C           & = 0.6\ln2                                                 \\
\end{align*}

При переходе (1) $\Rightarrow$ (2) надо проверить случай $y\equiv0$, но он не подходит в условие $y(2)=2$.

(3), (4) --- подстановка $x=2$.

Ответ: $-\ln|y| + 0.6\ln2 = \frac{1}{5}\left(\ln|x-3| - \ln|x+2|\right)$

\section{$2yy'''=y'$}

\begin{align*}
    2yy'''                   & = y'                    \\
    t := y'                  & \quad y''' = t''t^2+t't \\
    2y(t''t^2+t't)           & = t             \tag{1} \\
    2y(t''t+t')              & = 1                     \\
    2(t''t+t')               & = \frac{1}{y}           \\
    2(t''t+t') - \frac{1}{y} & = 0                     \\
\end{align*}

Очевидно $y\equiv\const$ --- решение из (1), других я не нашел

\section{$yy''+y'^2=1$}

\begin{align*}
    yy''+y'^2                                           & = 1                           \\
    t := y'                                             & \quad y'' = t'y               \\
    y^2t'+t^2                                           & = 1                           \\
    y^2t'+t^2                                           & = 1                           \\
    y^2t'                                               & = 1 - t^2             \tag{1} \\
    t'                                                  & = \frac{1 - t^2}{y^2} \tag{2} \\
    \frac{t'}{1 - t^2}                                  & = \frac{1}{y^2}               \\
    \int \frac{dt}{1 - t^2}                             & = \int \frac{dy}{y^2}         \\
    \int \frac{dt}{(1 - t)(1 + t)}                      & = -\frac{1}{y} + C            \\
    \int \frac{dt}{2(1 + t)} - \int \frac{dt}{2(1 - t)} & = -\frac{1}{y} + C            \\
    \frac{1}{2}(\ln|1+t| - \ln|1-t|)                    & = -\frac{1}{y} + C            \\
    \frac{1}{2}\ln\left|\frac{1+t}{1-t}\right|          & = -\frac{1}{y} + C            \\
    \frac{1}{2}\ln\left|\frac{1+y'}{1-y'}\right|        & = -\frac{1}{y} + C            \\
    \left|\frac{1+y'}{1-y'}\right|^{\frac{1}{2}}        & = e^{-\frac{1}{y}} \cdot e^C  \\
\end{align*}

$y\equiv0$ --- не решение, поэтому переход (1) $\Rightarrow$ (2) не теряет решений.

Не работает, попробуем $t = \cfrac{y'}{y}$:

\begin{align*}
    yy''+y'^2                                            & = 1                       \\
    t := \frac{y'}{y} \quad t' = \frac{yy'' - y'^2}{y^2} & \quad yy'' = t'y^2 + y'^2 \\
    t'y^2 + 2y'^2                                        & = 1                       \\
    t' + 2t^2                                            & = \frac{1}{y^2}           \\
\end{align*}

Это уравнение Рикатти, частное решение $t=\cfrac{1}{y}:$

\[-\frac{1}{y^2} + 2\frac{1}{y^2} = \frac{1}{y^2}\]

\begin{align*}
    t' + 2t^2                                                             & = \frac{1}{y^2}                                \\
    a := t - \frac{1}{y} \quad a' = t' + \frac{1}{y^2}                    & \quad a^2 = t^2 + \frac{1}{y^2} - \frac{2t}{y} \\
    a' - \frac{1}{y^2} + 2\left(a^2 - \frac{1}{y^2} + \frac{2t}{y}\right) & = \frac{1}{y^2}                                \\
    a' + a^2 + \frac{2t}{y}                                               & = \frac{2}{y^2}                                \\
    a' + a^2 + \frac{2\left(a + \frac{1}{y}\right)}{y}                    & = \frac{2}{y^2}                                \\
    a' + a^2 + \frac{2a}{y}                                               & = 0                                            \\
    a'                                                                    & = - a^2 - \frac{2a}{y} \tag{2}                 \\
    b := \frac{1}{a}                                                      & \quad b' = -\frac{a'}{a^2}                     \\
    \frac{a'}{a^2}                                                        & = -1 - \frac{2}{ay}                            \\
    -b'                                                                   & = -1 - \frac{2b}{y}                            \\
    b'                                                                    & = 1 + \frac{2b}{y}                             \\
    b' - \frac{2b}{y}                                                     & = 1                                            \\
    \frac{b'}{y^2} - \frac{2b}{y^3}                                       & = \frac{1}{y^2}                                \\
    \frac{b'}{y^2} - b(y^{-2})'                                           & = \frac{1}{y^2}                                \\
    \left(\frac{b}{y^2}\right)'                                           & = \frac{1}{y^2}                                \\
    \int \left(\frac{b}{y^2}\right)' dy                                   & = \int \frac{1}{y^2} dy                        \\
    \frac{1}{\left(t - \frac{1}{y}\right)y^2}                             & = -\frac{1}{y} + C                             \\
    \frac{1}{y^2\frac{y'}{y} - y}                                         & = -\frac{1}{y} + C                             \\
    \frac{1}{yy' - y}                                                     & = -\frac{1}{y} + C                             \\
    \frac{1}{y' - 1}                                                      & = -1 + yC                                      \\
    yC                                                                    & = y'(yC - 1)                                   \\
    \frac{yC}{yC - 1}                                                     & = y'                                           \\
    x                                                                     & = \int \frac{dy}{\frac{yC}{yC - 1}}            \\
    x                                                                     & = \int \frac{yC - 1}{yC}dy                     \\
    x                                                                     & = \int dy - \int \frac{1}{yC}dy                \\
    x                                                                     & = y - \frac{\ln|y|}{C} + C_1                   \\
\end{align*}

По (2) $a\equiv0$ подходит:

\begin{align*}
    t - \frac{1}{y}            & = 0     \\
    \frac{y'}{y} - \frac{1}{y} & = 0     \\
    y'                         & = 1     \\
    y                          & = x + C \\
\end{align*}

Ответ: $x = y - \frac{\ln|y|}{C} + C_1$ и $y = x + C$

\end{document}
