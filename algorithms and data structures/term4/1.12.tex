\input{preamble.sty}

\lhead{АиСД, задача 1.12}
\lfoot{Михайлов Максим}
\cfoot{}
\rfoot{M3237}

\begin{document}

\section*{Условие}

Есть поле \(n \times n\), некоторые клетки которого удалены. Положить на поле максимальное число доминошек \(2 \times 1\).

\section*{Решение}

Раскрасим поле в шахматном порядке. Представим клетки в виде вершин графа, где ребра идут только между соседними клетками.

\begin{figure}[h]
    \centering
    \includesvg{images/1.12.svg}
    \caption{Пример поля. Ребра --- серым.}
\end{figure}

Этот граф двудольный, т.к. никакие две клетки одного цвета не являются соседними, т.е. между ними нет рёбер. Найдём в этом графе максимальное паросочетание за \(\mathcal{O}(n^2 \cdot m) = \mathcal{O}(n^4)\). Заметим, что любому паросочетанию соответствует некоторое покрытие поля доминошками \(2 \times 1\) \textit{(каждая доминошка --- ребро)} и наоборот. Тогда максимальное паросочетание есть максимальное покрытие поля доминошками.

\end{document}