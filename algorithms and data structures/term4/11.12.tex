\input{preamble.sty}

\lhead{АиСД, задача 11.2}
\lfoot{Михайлов Максим}
\cfoot{}
\rfoot{M3237}

\begin{document}

\section{Условие}

Научитесь генерировать большие числа Кармайкла с тремя большими простыми делителями.

\section{Решение}

\subsection{Обозначения}

\begin{itemize}
    \item \(\mathcal{C}\) --- множество всех чисел Кармайкла.
    \item \(\mathcal{P}\) --- множество всех простых чисел.
\end{itemize}

\subsection{Теория}

\begin{wrapfigure}{R}{0.5\textwidth}
    \begin{center}
        \includegraphics[width=0.5\textwidth]{images/motivation.jpg}
        \caption{Кот, который мотивировал меня заниматься этой страшной задачей.}
    \end{center}
\end{wrapfigure}

\begin{theorem}\itemfix
    \begin{itemize}
        \item \(\forall i \ \ p_i \in \mathcal{P}\)
        \item \(n = \prod_i p_i\)
        \item \(\forall i \ \ p_i - 1 \equiv 0 \mod n - 1\)
    \end{itemize}
    Тогда \(n \in \mathcal{C}\). В обратную сторону также верно.
\end{theorem}
\begin{proof}
    \(\sphericalangle a : gcd(a, n) = 1\).

    По теореме Ферма:
    \begin{align*}
        a^{p_i - 1} & \equiv 1 \mod p_j   \\
        p_i - 1     & \equiv 0 \mod n - 1 \\
        a^{n - 1}   & \equiv 1 \mod p_j   \\
        a^{n - 1}   & \equiv 1 \mod n
    \end{align*}
\end{proof}

\begin{theorem}\itemfix
    \begin{itemize}
        \item \(p \in \mathcal{P}\)
        \item \(n = pu\)
    \end{itemize}

    Тогда \(p - 1 \equiv 0 \mod n - 1 \Leftrightarrow p - 1 \equiv 0 \mod u - 1\).
\end{theorem}
\begin{proof}
    \[(n - 1) - (u - 1) = n - u = pu - u = (p - 1)u\]
    Если \(p - 1\) делит ровно одно из \(\{n - 1, u - 1\}\), то левая часть не делится на \(p - 1\), а правая делится --- противоречие.
\end{proof}

\subsection{Первый метод}

Будем искать \(n \in \mathcal{C} : n = p \cdot q \cdot r, p < q < r\). По теореме 1:
\[\begin{cases}
        p - 1 \equiv 0 \mod n - 1 \\
        q - 1 \equiv 0 \mod n - 1 \\
        r - 1 \equiv 0 \mod n - 1 \\
    \end{cases}\]
По теореме 2:
\begin{equation}
    \begin{cases}
        p - 1 \equiv 0 \mod qr - 1 \\
        q - 1 \equiv 0 \mod pr - 1 \\
        r - 1 \equiv 0 \mod pq - 1
    \end{cases}
    \label{база}
\end{equation}

Если зафиксировать \(p, q\), то можно перебрать все \(d \equiv 0 \mod pq - 1\), такие что \(q < d\). Для каждого \(d\) нужно проверить, что \(d + 1 \in \mathcal{P}\). Если это так, третье условие выполнено.

\subsection{Второй метод}

Для \(p = 6N + 1, q = 12N + 1, r = 18N + 1\) условие \eqref{база} выполнено:
\[\begin{cases}
        6N \equiv 0 \mod 216 N^2 + 30N   \\
        12N \equiv 0 \mod 108 N^2 + 24 N \\
        18N \equiv 0 \mod 72 N^2 + 18 N
    \end{cases} \]
и нужно только проверять на простоту. Небольшой трюк: если для \(\tilde{N}\) \(p\) не простое, то \(2\tilde{N}\) и \(3 \tilde{N}\) рассматривать не нужно.

\section{Анализ}

\subsection{Первый метод}

Генерировать все \(p, q < n : p,q\in \mathcal{P}\) занимает \(\mathcal{O}(n / \log \log n)\) и асимптотически мы найдём \(\frac{n^2}{\log^2 n}\) пар, т.е.
\[\frac{\log^2 n}{n^2} \cdot \frac{n}{\log \log n} = \frac{\log^2 n}{n \log \log n}\]
итераций на пару. Я не придумал, как оценить число делителей \(pq - 1\), поэтому дальше не анализируется.

\subsection{Второй метод}

Проверить на простоту детерминированно занимает \(\mathcal{O}(\log^6 n)\) с помощью AKS. Т.к. вероятность случайно выбрать простое число есть \(\Theta\left(\frac{1}{\log n}\right)\), то вероятность выбрать 3 простых числа --- \(\Theta\left(\frac{1}{\log^3 n}\right)\). Таким образом, получается \(3 \cdot \mathcal{O}(\log^6 n) \cdot \Theta(\log^3 n) = \mathcal{O}(\log^9 n)\).

Ещё можно использовать модификацию AKS за \(\mathcal{O}(\log^3 n)\), которая работает, если гипотеза Агравала верна, и если мы выдадим неверный ответ, то заявляем всем, что мы её опровергли :).

% \subsection{Третий метод}

% Запишем \eqref{база} в другом виде:

% \[
%     \begin{cases}
%         a(p - 1) = qr - 1 \\
%         b(q - 1) = pr - 1 \\
%         c(r - 1) = pq - 1
%     \end{cases}
% \]

% Пусть \(\delta = bc - q^2\). При фиксированных \(p, с, \delta\) однозначно задается \(q\):

% \begin{theorem}
%     \[q - 1 = \frac{(p - 1)(p + c)}{\delta}\]
% \end{theorem}
% \begin{proof}
%     \begin{align*}
%         q - 1             & = \frac{pr - 1}{b}                        \\
%                           & = \frac{pr - 1 + p - p}{b}                \\
%                           & = \frac{p(r - 1) + (p - 1)}{b}            \\
%                           & = \frac{\frac{p}{c}(pq - 1) + (p - 1)}{b} \\
%                           & = \frac{\frac{p}{c}(pq - 1) + (p - 1)}{b} \\
%                           & = \frac{p(pq - 1) + c(p - 1)}{bc}         \\
%         bc(q - 1)         & = p(pq - 1) + c(p - 1)                    \\
%                           & = p^2q - p + c(p - 1)                     \\
%                           & = p^2q - p - p^2 + p^2 + c(p - 1)         \\
%                           & = p^2(q - 1) - p + p^2 + c(p - 1)         \\
%         (bc - p^2)(q - 1) & = p(p - 1) + c(p - 1)                     \\
%         q - 1             & = \frac{(p + c)(p - 1)}{bc - p^2}         \\
%     \end{align*}
% \end{proof}

% И следовательно однозначно задается \(r\).

% Казалось бы, 

\end{document}
