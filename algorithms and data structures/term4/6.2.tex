\input{preamble.sty}

\lhead{АиСД, задача 6.2}
\lfoot{Михайлов Максим}
\cfoot{}
\rfoot{M3237}

\begin{document}

\section*{Условие}

Покажите, что если убрать требование о целочисленности кососимметрического потока, то ее
можно решать, построив обычный поток, после чего перестроив его в кососимметрический
поток такого же размера.

\section*{Решение}

Рассмотрим кососимметрический граф \(G\), в котором мы построили \textit{обычный} максимальный поток \(f\) значения \(F\). Единственное, что ему не хватает для кососимметричности --- выполнение условия \(f_{uv'} = f_{vu'}\). Если мы построим поток значения \(F\), выполняющий это условие, то он будет максимальным, т.к. потока значения \( > F\) нет.

Построим поток \(f'\), где \(f'_{uv'} = f'_{vu'} = \cfrac{f_{uv'} + f_{vu'}}{2}\). Условие кососимметричности выполнено, докажем, что это всё ещё поток.

Рассмотрим вершину \(v\). При переходе \(f \to f'\) в вершине \(v\) разность входящего и выходящего потока изменилась на:

\begin{align*}
    \Delta v & = \sum_{u : \exists uv\in E} \left(\frac{f_{uv} + f_{u'v'}}{2} - f_{uv}\right) - \sum_{w : \exists vw\in E} \left(\frac{f_{vw} + f_{v'w'}}{2} - f_{vw}\right) \\
             & = \sum_{u : \exists uv\in E} \frac{f_{u'v'} - f_{uv}}{2} + \sum_{w : \exists vw\in E} \frac{f_{vw} - f_{v'w'}}{2}                                             \\
             & = \sum_{u, w} \frac{f_{u'v'} - f_{v'w'}}{2} - \sum_{u, w} \frac{f_{uv} - f_{vw}}{2}                                                                           \\
             & = \frac{1}{2} \mathfrak{F}(v') - \frac{1}{2} \mathfrak{F}(v)                                                                                                  \\
             & = 0
\end{align*}

, где \(\mathfrak{F}(a)\) обозначает разность входящего и выходящего потока в вершине \(a\) в оригинальном потоке, т.е. \(0\).

\begin{figure}[h]
    \centering
    \includegraphics[scale=0.35]{images/kot.jpg}
    \caption{\sout{Косо}симметричный кот}
\end{figure}

\end{document}