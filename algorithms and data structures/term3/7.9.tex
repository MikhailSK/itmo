\input{preamble.sty}

\lhead{АиСД, задача 7.9}
\lfoot{Михайлов Максим}
\cfoot{}
\rfoot{M3237}

\begin{document}

\section{Условие}

Пусть в задаче о минимальном остовном дереве добавлено следующее ограничение: для каждой
вершины $v$ задано значение $d[v]$. Требуется найти минимальное остовное дерево, в котором
степени вершин не превосходят соответствующих значений $d[v]$. Покажите, что эта задача не
проще, чем задача о нахождении минимального гамильтонова пути (которая, как известно,
NP-полна).

\section{Решение}

Решим задачу о нахождении минимального гамильтонова пути через решение исходной
задачи.

Пусть надо найти минимальный гамильтонов путь $u\rightsquigarrow w$. Тогда $d[v]
    = \begin{cases}
        1 , & v = u        \\
        1 , & v = w        \\
        2 , & \text{иначе}
    \end{cases}$

Решение исходной задачи при таких входных данных даст минимальный бамбук с
``краями'' $u$ и $w$, который покрывает все вершины графа. Это и есть
минимальный гамильтонов пути $u\rightsquigarrow w$.

\begin{remark}
    Получится именно бамбук, т.к. если есть разветвление, то степень некоторой
    вершины $> 2$.
\end{remark}

\begin{figure}[h]
    \includegraphics{images/kot_bambuk.jpg}
    \centering
    \caption{Кот ест бамбук, потому что он ветвится и ломает теорию графов}
\end{figure}

\end{document}