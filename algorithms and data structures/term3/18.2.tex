\input{preamble.sty}

\usepackage{svg}
\usepackage{listings}

% \usepackage{inconsolata}
\lstset{basicstyle=\ttfamily,breaklines=true}

\setmonofont{Inconsolata}

\lhead{АиСД, задача 18.2}
\lfoot{Михайлов Максим}
\cfoot{}
\rfoot{M3237}

\begin{document}

\section*{Условие}

Отвечать на запросы: входит ли строка \(s\) в строку \(k\cdot t\) (k повторений строки t). Предподсчет за \(\mathcal{O}(|t|)\), ответ за \(\mathcal{O}(|s|)\).

\section*{Решение}

Преподсчитаем сжатое суффиксное дерево для строки \(2t\). Если \(|s| < t\), то эта задача --- просто проверка, что \(s\) подходит в дерево для \(2t\).

Пусть \(|s| \ge t\). Тогда пусть префикс \(s\) длины \(t\) это строка \(f\).

\(s\) входит в \(k\cdot t \iff s = m\cdot f + \text{pref}f\), где \(\text{pref}f\) --- некоторый префикс \(f\), см. рисунок.

\begin{figure}[h]
    \centering
    \includesvg[scale=1.5]{images/ksf.svg}
    \caption{Отсюда очевидно, что \(s = m\cdot f + \text{pref} f\)}
\end{figure}

Заметим, что \(f\) это некоторая подстрока \(2\cdot t\), поэтому мы можем его найти с помощью построенного суффиксного дерева. Спустимся по \(s\) в дереве на \(|t|\) символов. Пройденный путь есть \(f\). Если мы не смогли пройти, то ответ на запрос --- нет. Теперь, когда известен \(f\), в линию пройдём по \(s\) и проверим, что \(s = m\cdot f + \text{pref} f\).

\begin{lstlisting}[language=Python]
for i in 0..len(s):
    if s[i] != f[i % len(f)]:
        return false
return true
\end{lstlisting}

Кроме того, необходимо проверить, что \(s\) не выходит за пределы \(k\cdot t\), т.е. \(\delta + |s| \le k\cdot |t|\). \(\delta\) можно тривиально получить при нахождении \(f\).

\begin{figure}[h]
    \includegraphics[scale=0.2]{images/kot_zabor.jpg}
    \centering
    \caption{Котёнок идет в линию по строке}
\end{figure}


\end{document}