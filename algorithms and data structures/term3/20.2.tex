\input{preamble.sty}

\usepackage{listings}

% \usepackage{inconsolata}
\lstset{basicstyle=\ttfamily,breaklines=true}

\setmonofont{Inconsolata}

\lhead{АиСД, задача 20.2}
\lfoot{Михайлов Максим}
\cfoot{}
\rfoot{M3237}

\begin{document}

\section{Условие}

Дано ориентированное (подвешенное) дерево из n вершин, на каждом ребре написан символ. Найти число путей, идущих вниз по дереву, соответствующих строке $s$. Алфавит константного размера, время \(\mathcal{O}(n + |s|)\).

\section{Решение}

Заметим, что каждый путь от корня до листа задает некоторую строку. Обозначим множество всех таких строк \(A\). Тогда задача сводится к нахождению количества вхождений \(s\) во все строки \(t\in A\).

Рассмотрим быстрый (\(\mathcal{O}(|s|)\)) алгоритм построения префикс-функции.

\begin{lstlisting}[language=Python]
p[0] = 0
for i = 1 to s.length - 1
    k = p[i - 1]
    while k > 0 and s[i] != s[k]
        k = p[k - 1]
    if s[i] == s[k]
        k++
    p[i] = k
\end{lstlisting}

Мы можем применить этот алгоритм ко всему дереву, т.к. алгоритм либо поднимается вверх ( \texttt{k = p[k - 1]} ), либо спускается на единицу ( \texttt{k++} ). Спуск вниз можно реализовать наподобие dfs --- спускаемся во всех детей по очереди. Но подъем вверх \textit{(на произвольную высоту)} необходимо делать за константу, что нетривиально.

Задача подъема в дереве --- Level ancestor problem и она решается в предподсчёт \(\mathcal{O}(n)\), запрос \(\mathcal{O}(1)\). Общая идея состоит в том, чтобы разложить дерево по четырём русским, потом разложить его в непересекающиеся длинные пути, потом увеличить длину путей вверх в два раза и двоичными подъемами найти ответ для каждого пути.

Таким образом, мы можем посчитать префикс-функцию для всего дерева. Итоговый алгоритм следующий:
\begin{enumerate}
    \item Подвесим корень дерева к пути, построенному из строки \(s\#\)
    \item Посчитаем префикс-функцию для дерева
    \item Обойдём дерево и посчитаем количество вершин \(v\), таких что \(p[v] = n\)
\end{enumerate}



\end{document}