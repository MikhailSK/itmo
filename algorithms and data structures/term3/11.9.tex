\input{preamble.sty}

\usepackage{listings}

% \usepackage{inconsolata}
\lstset{basicstyle=\ttfamily\footnotesize,breaklines=true}

\setmonofont{Inconsolata}

\lhead{АиСД, задача 11.9}
\lfoot{Михайлов Максим}
\cfoot{}
\rfoot{M3237}

\begin{document}

\section{Условие}

Дан взвешенный граф. Удалить максимальное число ребер, при условии что расстояние от $s$ до $t$ должно быть не больше $d$.

\section{Решение}

Найдём путь $s\to t$ с расстоянием $\le d$ и минимальным числом ребер. Тогда ответ --- удалить все ребра, кроме ребер этого пути.

Алгоритм Форда-Беллмана считает динамику $d[u][k]$ --- минимальный вес пути $s\to u$ с $\le k$ ребер. Тогда мы можем найти искомый путь $\le d$ следующим образом:

\begin{lstlisting}[language=Python]
dp = ford-bellman()
for k in 0 .. n-2:
    if dp[t][k] <= d:
        return k
\end{lstlisting}

После нахождения \texttt{k} искомый путь находится обратной динамикой.

\end{document}