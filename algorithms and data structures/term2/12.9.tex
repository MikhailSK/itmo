\documentclass[12pt, a4paper]{article}
\usepackage[left=2cm, right=2cm, top=2cm, bottom=1.5cm, bindingoffset=0cm, headheight=15pt]{geometry}

\usepackage{fancyhdr}
\usepackage{amsmath}
\usepackage{amsthm}
\usepackage{listings}
\usepackage{xcolor}
\usepackage[T2A]{fontenc}
\usepackage[utf8]{inputenc}
\usepackage[english,russian]{babel}
\usepackage{graphicx}

% \setmainfont{Linux Libertine}

\renewcommand{\thesubsection}{\arabic{subsection}.}
\makeatletter
\renewcommand*{\@seccntformat}[1]{\csname the#1\endcsname\hspace{0.1cm}}
\makeatother

\pagestyle{fancy}
\lhead{}
\chead{Алгоритмы и структуры данных, задача 12.9}
\cfoot{Михайлов Максим, М3137}

\definecolor{mygreen}{rgb}{0,0.6,0}

\lstdefinestyle{customc}{
  belowcaptionskip=1\baselineskip,
  breaklines=true,
%   frame=L,
  xleftmargin=\parindent,
  language=C,
  showstringspaces=false,
  basicstyle=\ttfamily,
  keywordstyle=\color{blue},
%   keywordstyle=\bfseries\color{green!40!black},
  commentstyle=\color{mygreen},
%   identifierstyle=\color{blue},
  stringstyle=\color{orange},                 
  numbers=left,               
  numbersep=7pt,
}
\lstset{escapechar=@,style=customc}

\begin{document}
    
\section*{Условие}

Дано дерево, над ним проводят секретные эксперименты. Каждый эксперимент $(v, d)$ состоит
в том, что в вершине v распыляют вонючее вещество с вонючестью $d$. При этом для всех $x < d$
уровень вони во всех вершинах на расстоянии $x$ увеличивается на $d - x$. Отвечать на запросы:
провести эксперимент, найти текущий уровень вони в вершине.

\section*{Решение}

Заметим, что эта задача похожа на 12.7, поэтому тоже будем использовать центроидную декомпозицию с ДО и сортированными массивами.

Разница с 12.7:
\begin{enumerate}
    \item Вычитаем из $d$ расстояние до следующего центроида, когда поднимаемся в него.
    \item Когда запрос уходит из вершины (происходит \texttt{propagate}), в эту вершину прибавляется $d-l$, где $l$ --- расстояние до центроида, $d$ --- значение в запросе.
    \item Чтобы избежать многократного увеличения значения в какой-либо вершине при одном запросе, будем передавать вместе с запросом номер этого запроса, также будем хранить номер последнего обработанного запроса в вершине. Тогда если в вершину приходит запрос, номер которого совпадает с последним обработанным, она его игнорит.
\end{enumerate}

\textbf{Асимптотика} : такая же, как в 12.7, т.е. $\mathcal O(\log^2 n)$, пока не вырождаемся в длинную арифметику с подсчетом номера запроса.

\end{document}