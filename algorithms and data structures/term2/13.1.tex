\documentclass[12pt, a4paper]{article}
\usepackage[left=2cm, right=2cm, top=2cm, bottom=1.5cm, bindingoffset=0cm, headheight=15pt]{geometry}

\usepackage{fancyhdr}
\usepackage{amsmath}
\usepackage{amsthm}
\usepackage{listings}
\usepackage{xcolor}
\usepackage[T2A]{fontenc}
\usepackage[utf8]{inputenc}
\usepackage[english,russian]{babel}
\usepackage{graphicx}

% \setmainfont{Linux Libertine}

\renewcommand{\thesubsection}{\arabic{subsection}.}
\makeatletter
\renewcommand*{\@seccntformat}[1]{\csname the#1\endcsname\hspace{0.1cm}}
\makeatother

\pagestyle{fancy}
\lhead{}
\chead{Алгоритмы и структуры данных, задача 13.1}
\cfoot{Михайлов Максим, М3137}

\definecolor{mygreen}{rgb}{0,0.6,0}

\lstdefinestyle{customc}{
  belowcaptionskip=1\baselineskip,
  breaklines=true,
%   frame=L,
  xleftmargin=\parindent,
  language=python,
  showstringspaces=false,
  basicstyle=\ttfamily,
  keywordstyle=\color{blue},
%   keywordstyle=\bfseries\color{green!40!black},
  commentstyle=\color{mygreen},
%   identifierstyle=\color{blue},
  stringstyle=\color{orange},                 
  numbers=left,               
  numbersep=7pt,
}
\lstset{escapechar=@,style=customc}

\begin{document}

\section*{Условие}

Сделать стек во внешней памяти. Время работы операций $\mathcal O(1/B)$.

\section*{Решение}

Будем хранить стек из $2B$ элементов (\texttt{st}) начала полного стека во внутренней памяти.

\begin{lstlisting}
pop():
  if st.size == 0
    load(st)
  st.pop()

push(x):
  if st.size == 2 * B
    unload(st[0:B])
  st.push(x)
\end{lstlisting}

Асимптотика $\mathcal O(1/B)$, потому что между операциями чтения/записи есть хотя бы $B$ вызовов \texttt{push/pop}. Это так, потому что после каждой IO операции размер стека в памяти $B$, а следующая операция произойдет только когда размер стека $0$ или $2B$, чего можно достичь в $\ge B$ штук \texttt{push/pop}.

\begin{figure}[b]
  \includegraphics[scale=0.4]{cat.jpg}
  \centering
\end{figure}

\end{document}