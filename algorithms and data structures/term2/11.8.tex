\documentclass[11pt, a4paper]{article}
\usepackage[left=2cm, right=2cm, top=2cm, bottom=1.5cm, bindingoffset=0cm, headheight=15pt]{geometry}

\usepackage{fancyhdr}
\usepackage{amsmath}
\usepackage{amsthm}
\usepackage{listings}
\usepackage{xcolor}
\usepackage[T2A]{fontenc}
\usepackage[utf8]{inputenc}
\usepackage[english,russian]{babel}
\usepackage{graphicx}

% \setmainfont{Linux Libertine}

\renewcommand{\thesubsection}{\arabic{subsection}.}
\makeatletter
\renewcommand*{\@seccntformat}[1]{\csname the#1\endcsname\hspace{0.1cm}}
\makeatother

\pagestyle{fancy}
\lhead{}
\chead{Алгоритмы и структуры данных, задача 11.8}
\cfoot{Михайлов Максим, М3137}

\definecolor{mygreen}{rgb}{0,0.6,0}

\lstdefinestyle{customc}{
  belowcaptionskip=1\baselineskip,
  breaklines=true,
%   frame=L,
  xleftmargin=\parindent,
  language=C,
  showstringspaces=false,
  basicstyle=\ttfamily,
  keywordstyle=\color{blue},
%   keywordstyle=\bfseries\color{green!40!black},
  commentstyle=\color{mygreen},
%   identifierstyle=\color{blue},
  stringstyle=\color{orange},                 
  numbers=left,               
  numbersep=7pt,
}
\lstset{escapechar=@,style=customc}

\begin{document}

\section*{Условие}

Пусть в каждой компоненте связности выполняется условие: $E - V \leq C$, где $E$ — число ребер,
$V$ — число вершин, $C$ — небольшая константа. Как работать с такими графами?

\section*{Решение}

Из условия следует, что в каждой компоненте связности могут быть циклы, но их мало ($C$). Таким образом, эта задача --- более общий случай 11.7, поэтому сделаем все то же самое, как в 11.7, но будем хранить в корне каждой компоненты не одно фейковое ребро, а вектор из $\leq C$ этих ребер.

\subsection{\texttt{cut}}
Есть три случая:

\begin{enumerate}
    \item \textit{Режем фейковое ребро}: удаляем его из вектора.
    \item \textit{Режем настоящее ребро, фейковых ребер нет в этой КС}: обычный \texttt{cut}.
    \item \textit{Режем настоящее ребро, есть фейковые ребра}: обычный \texttt{cut}, потом перебираем все фейковые ребра из разрезанной КС и если некоторое из них соединяет вершины в разных КС, то по этому ребру происходит \texttt{link}. Если таких ребер нет, то надо все фейковые ребра раскидать по новым двум КС.

          Для случая 3:
          \begin{lstlisting}
fakes = root(u).fakes
vanilla_cut(u, v)
for (edge : fakes)
    if (!connected(edge.first. edge.second))
        link(edge.first. edge.second)
        break
else // python-style, if no "break"
    for (edge : fakes)
        if (connected(edge.first, u))
            u.parent.fakes.push_back(edge)
        else
            v.parent.fakes.push_back(edge)
    \end{lstlisting}
\end{enumerate}

\subsection{\texttt{link}}
Если линкуемые вершины лежат в одной и той же компоненте связности, то добавляем фейковое ребро в вектор, лежащий в корне этой КС. Иначе делаем обычный \texttt{link} и сливаем вектора фейковых ребер.

\subsection{\texttt{connected}}
Циклы не изменяют связность, поэтому эта операция не изменилась.

\begin{figure}[h]
    \includegraphics[scale=0.35]{kot.png}
    \centering
    % \caption{
    %     \textit{кот}: я с 7 лабами за месяц;
    %     \textit{масло}: +3 балла по алгосам
    % }
\end{figure}

\end{document}
