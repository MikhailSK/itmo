\documentclass[12pt, a4paper]{article}
\usepackage[left=2cm, right=2cm, top=2cm, bottom=1.5cm, bindingoffset=0cm, headheight=15pt]{geometry}

\usepackage{fancyhdr}
\usepackage{amsmath}
\usepackage{amsthm}
\usepackage{listings}
\usepackage{xcolor}
\usepackage[T2A]{fontenc}
\usepackage[utf8]{inputenc}
\usepackage[english,russian]{babel}
\usepackage{graphicx}

% \setmainfont{Linux Libertine}

\renewcommand{\thesubsection}{\arabic{subsection}.}
\makeatletter
\renewcommand*{\@seccntformat}[1]{\csname the#1\endcsname\hspace{0.1cm}}
\makeatother

\pagestyle{fancy}
\lhead{}
\chead{Алгоритмы и структуры данных, задача 12.7}
\cfoot{Михайлов Максим, М3137}

\definecolor{mygreen}{rgb}{0,0.6,0}

\lstdefinestyle{customc}{
  belowcaptionskip=1\baselineskip,
  breaklines=true,
%   frame=L,
  xleftmargin=\parindent,
  language=C,
  showstringspaces=false,
  basicstyle=\ttfamily,
  keywordstyle=\color{blue},
%   keywordstyle=\bfseries\color{green!40!black},
  commentstyle=\color{mygreen},
%   identifierstyle=\color{blue},
  stringstyle=\color{orange},                 
  numbers=left,               
  numbersep=7pt,
}
\lstset{escapechar=@,style=customc}

\begin{document}
    
\section*{Условие}

Дано дерево, вершины которого можно красить. Отвечать на запросы:
\begin{itemize}
    \item по заданным $(v, d, c)$ покрасить все вершины на расстоянии не больше $d$ от вершины $v$ в цвет $c$
    \item узнать, какой сейчас цвет у вершины $v$
\end{itemize}

\section*{Решение}

Эта задача похожа на задачу, разобранную на лекции \textit{(нахождение числа вершин на расстоянии от некоторой)}, но в структуре данных, которая заменяет сортированные массивы, нужно уметь делать присвоение на префиксе \textit{(покраска)}. Это умеет делать дерево отрезков, воспользуемся им.

Разобьем дерево на центроидную декомпозицию, преподсчитаем расстояния до центроидов. Каждому центроиду дадим сортированный массив расстояний всех его детей \textit{(внуков, и т.д.)}, а также дадим дерево отрезков, где будут лежать цвета всех вершин в таком же порядке, как эти вершины лежат в сортнутом массиве.

Запрос покраски в поддереве центроида выполняется так: \texttt{tree.set(0, i)}, где \texttt{i} - индекс самой далекой от центроида вершины, такой что $d_i \le d$. Этот индекс находится в массиве бинпоиском, как и в исходной задаче.

Ленивый \texttt{propagate} дает $\mathcal O (\log n)$, все остальные операции - тоже полилог.


\end{document}